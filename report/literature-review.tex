This literature review will explore existing research and approaches to other combinatorial optimisation problems.
In this review existing research for the Travelling Salesman Problem, Multiple Travelling Salesman
Problem, Vehicle Routing Problem and Tourist Trip Design Problem, will be covered.
By gaining an understanding of the strengths and limitations of existing approaches to similar problems, better informed
decisions can be made regarding which approaches to investigate, how they may be adapted to suit specific constraints of
the Multi-Day Trip Planning Problem, and how these approaches might be implemented in practice.
While the approaches taken to these problems may not be directly applicable to this study, it is likely their
techniques can be adapted to better suit the problem investigated in this report.\\

\noindent
The Travelling Salesman Problem (TSP) is perhaps one of the most studied combinatorial optimisation problems in
computer science.
The extensive research on the problem has acted as an `engine of discovery for general-purpose techniques' offering
large contributions across a wide range of mathematics\parencite[p. 40--41]{applegate2006traveling}.
The TSP can be described simply as: Given a set of locations and the cost of travel between them, find the shortest
route that visits each location and returns to the start~\parencite[p. 1]{applegate2006traveling}.
The problem relates directly to this goal of minimising travel time considered in this study, in fact, for inputs where
$d = 1$, the Multi-Day Trip Planning problem becomes the TSP, with greater values of $d$ introducing additional
complexity.

The TSP is proven to be NP-hard~\parencite[p. 1096--1097]{cormen2022introduction}, meaning that there
are no known algorithms capable of solving the problem in polynomial time.
Exact methods, such as brute force, branch-and-bound and dynamic programming are capable of finding optimal
solutions, they just take an impractically long time to do so.
\textcite[p. 489--530]{applegate2006traveling} discusses how the best solvers of the time had solved problems of
thousands of locations, but took many CPU years to do so.
Even with advances in computer processing, finding exact solutions appear too impractical in the context of this study.
This leads towards the investigation of heuristic and approximation algorithms, which aim to find a near optimal
solution within a reasonable time frame.

\textcite{laporte1992traveling} categorises TSP heuristics as either constructive or improvement heuristics, and
claims that the best approaches to the problem are a combination of both.
The paper presents a number of heuristic algorithms across these three approaches including: nearest neighbour
routing, which iteratively adds the nearest location to a route; r-opt swapping, which takes an existing route,
removes r connections and rebuilds the tour optimally; and the CCAO algorithm, which uses a convex hull and cheapest
insertion procedure to build a route, before improving it via angle maximisation and r-opt swapping.
While the various algorithms presented in the paper are worthy of note, there is a lack of significant discussion
regarding the performance of these algorithms, and the comparisons they do make, lack supporting evidence.

A recent study by~\textcite{goutham2023convex} investigated performance of the Convex Hull and Cheapest Insertion (CHCI)
steps of the CCAO algorithm for travelling salesman problems in non-Euclidean space.
The study used existing TSP datasets and modified them to add separators to the graph, or by using the $\mathcal{L}_1$
norm to calculate distances, however the convex hull was still formed according to the initial Euclidean graph.
Despite this, CHCI was shown to outperform other heuristic and meta-heuristic approaches, such as Nearest Insertion or
Ant Colony Optimisation, the majority of the time.
While the non-Euclidean spaces considered differ to the input space used in this study, in which graphs are asymmetric
and based on travel time between locations, there is potential to utilise Euclidean heuristics as a starting point for
finding routes.\\

\noindent
The multiple travelling salesman problem (mTSP) is an extension of the TSP, this time aiming to find a set of $m$
routes that together visit every location once, with the goal of minimising the total cost of all
routes~\parencite[p. 209]{bektas2006multiple}.
This comes closer to the focus of this study, in which several routes are sought over multiple days, but without the
optimisation objective of balancing routes.
A literature review by~\textcite{bektas2006multiple} provides a comprehensive overview of procedures for solving the
mTSP, of particular interest are~\textcite{bellmore1974transformation}'s transformation of the mTSP into the TSP,
and~\textcite{tang2000multiple}'s Modified Genetic Algorithm solution.

By transforming the mTSP into a TSP, one is able to use existing TSP algorithms to find mTSP routes.
If similar transformations could be applied within this study, the possibility exists to take advantage of existing TSP
research to implement algorithms that are already known to be effective.
\textcite{bellmore1974transformation} propose a method of modifying a TSP graph to include artificial nodes and
edges, with the nodes indicating the end of one salesman's route and the start of another and the edges including
the cost of including additional salesmen in the problem.

\textcite{tang2000multiple} modelled the schedulling of steel production as an mTSP before performing a similar
transformation to \textcite{bellmore1974transformation}'s, to convert their mTSP into a TSP and solve the problem
using their modified genetic algorithm (MGA).
Typically at the time, genetic algorithms used a selection procedure in which both parents were chosen semi-randomly
from the population, with each individuals fitness increasing their chance to be chosen.
\textcite{tang2000multiple}'s MGA altered this selection so one of the parents would always be the best individual,
helping to maintain good solutions and reach convergence quicker.
Not only did this MGA approach prove effective in computational testing, but when employed for scheduling in an iron
and steel complex in Shanghai, showed an average improvement of 20\%~\parencite[p. 278--281]{tang2000multiple}.

Unfortunately, considering how the objective function of this study differs from most other combinatorial optimisation
problems, it is likely that performing such a transformation could overly complicate the resultant problem.
Nevertheless, the effectiveness of applying TSP solvers to other problems, related or not is certainly interesting.
It would be worth considering scenarios that would allow the utilisation of pre-existing algorithms without too much
modification.\\

\noindent
First introduced by \textcite{dantzig1959truck}, the Vehicle Routing Problem (VRP) is similar to the mTSP, considering
multiple vehicles departing from, and returning to, a central depot while visiting a number of locations.
The VRP typically extends the problem to include a number of constraints related to the context of servicing
customers, which does bring it to differ from the problem being studied.

One approach of particular interest is the generalised assignment heuristic considered by~\textcite{fisher1981generalized},
which divided the input space geometrically, assigning each vehicle to one of these divisions.
After these locations were assigned to a vehicle a route visiting them was found and optimised to reduce travel time.
The generalised heuristic was shown to outperform the methods it was compared against~\parencite[p. 123]{fisher1981generalized}.

Further investigation into this approach has been carried out by those such as~\textcite{nallusamy2010optimization}, who
combined the use of both k-means clustering and genetic algorithms to approximate solutions to the mVRP.
K-Means was used to group nearby locations and have them visited by the same vehicle, then genetic algorithms were
used to find the routes of each vehicle independently.
These approaches seem particularly relevant to this study's problem, locations could be assigned to given days of the
overall trip.\\

\noindent
The Tourist Trip Design Problem (TTDP) takes potential points of interest (PoIs) and attempts to find the most
interesting route based on a number of criteria.
The TTDP shares the same motivation of tourist route planning as the problem investigated in this study and often
considers similar parameters, such as travel time between PoIs and the desired time spent at each one~\parencite{vansteenwegen2007mobile}.
A large difference though, is that it does not require every location to be visited, prioritising the visitation of
certain PoIs according to user preference.

\textcite{vansteenwegen2007mobile} discusses modelling multiple day trip planning through the Team Orienteering
Problem (TOP).
The TOP aims to find a set of routes that visit a number of locations, with the goal of maximising the total value of
the locations visited within some time limit.
\textcite{vansteenwegen2011orienteering}'s survey provides a summary of the best TOP algorithms according
to benchmarks on their speed and solution quality.
Among the best performing algorithms were meta-heuristics such as Genetic Based Tabu Search, Ant Colony Optimisation
and Greedy Randomised Adaptive Search Procedure.
The paper also discusses how different local search algorithms apply different `local search moves' to improve the
quality of a solution.
While a wide range of moves exist, all effective search algorithms used greedy insertion as part of their
solution, perhaps indicating its effectiveness.\\

\noindent
The literature reviewed in this section provides valuable insights into how the problem may be tackled.
With the expansive pre-existing research into the TSP, mTSP, VRP and TTDP, existing
approaches and how they may be applicable for this project, may be evaluated.
A diverse range of approximate solutions are considered in the reviewed literature, from simple construction
heuristics such as nearest neighbour routing or greedy insertion, to complex meta-heuristics such as genetic
algorithms or ant colony optimisation.

Among the most promising approaches are those that combine both clustering and routing algorithms to split up locations
and find multiple routes from these divisions.
It is unfortunate that unlike this study's problem, the literature covering these techniques are only concerned with
minimising overall route length, hopefully these methods will still prove useful despite these differences.

Much of the research reviewed does not consider the optimisation of multiple objectives, with none of
it sharing the goals of this study to minimise both travel and route variance, this would suggest that more research is
needed in this area.\\

\noindent
Considering the lack of investigation into solving problems with the constraints and optimisation objectives applied in
this study, this project will compare a swathe of approaches and algorithms, aiming to identify which are most
appropriate for the problem at hand.

The majority of approaches will be those that split routing and assignment into two steps, which will be performed
in one of two ways:
\begin{itemize}
    \item Assigning locations to a given day in the trip and then finding routes for each group, as previously
    described by~\textcite{fisher1981generalized} and~\textcite{nallusamy2010optimization}.
    \item Finding a route between all locations and then using insertion algorithms, such as those compared by~\textcite{vansteenwegen2011orienteering},
    to break the route up into a multiple day trip.
\end{itemize}
\noindent
Multiple routing and clustering algorithms will be implemented for use in these approaches, as will be described in
the next section of this report.