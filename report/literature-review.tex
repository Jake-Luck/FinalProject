This literature review aims to explore existing research and approaches to other combinatorial optimization problems.
There is extensive previous research on various combinatorial problems, for example, the Travelling Salesman
Problem (TSP), Vehicle Routing Problem (VRP) and Tourist Trip Design Problem (TTDP).
It is important to understand how these problems are similar to the one presented in this report, as well as where
those similarities end.
By gaining an understanding of the strengths and limitations of existing approaches to similar problem, we can make
better informed decisions regarding which approaches to investigate, how they may be adapted to suit our specific
constraints and how they might be implemented in practice.
While the approaches taken to these problems may not be directly applicable to our own, it is likely we can adapt their
techniques to suit the specific constraints of this problem.

[paper1] defines [problem1] as [definition].
blah blah blah.
This is similar to our problem because [reason].
Another problem is [problem1], which is defined by [paper2] as [definition].
blah blah blah.
This is similar to our problem because [reason].

[paper3] proposes a solution to [problem] using [approach].
blah blah blah, pros, cons, etc.

READ, YOU NEED TO READ THE PAPERS stop faffing about and read the papers


\subsection{Related Optimization Problems}
\begin{itemize}
    \item Vehicle Routing Problems (VRP), particularly Multiple-Trip VRP (Cattaruzza et al., 2016)
    \item Traveling Salesman Problem variants (Applegate et al., 2006)
    \item Tourist Trip Design Problem (TTDP) (Vansteenwegen et al., 2011; Gavalas et al., 2014)
    \item Orienteering Problems with time constraints (Gunawan et al., 2016)
\end{itemize}

\subsection{Algorithms for Multi-Visit Routing}
\begin{itemize}
    \item Exact methods: Branch and Bound, Dynamic Programming, Integer Linear Programming (Laporte, 1992)
    \item Heuristic approaches for multi-visit scenarios (Necula et al., 2015)
    \item Metaheuristics: Genetic Algorithms (Potvin, 1996), Simulated Annealing (Černý, 1985), Ant Colony Optimization (Dorigo \& Gambardella, 1997)
    \item Approaches handling returning to a depot multiple times (Azi et al., 2014)
\end{itemize}

\subsection{Multi-Objective Optimization Approaches}
\begin{itemize}
    \item Survey of approaches for handling multiple objectives
    \begin{itemize}
        \item Weighted sum methods (Marler \& Arora, 2010)
        \item Pareto optimization approaches (García-Nájera et al., 2013)
        \item Constraint-based methods (Konak et al., 2006)
    \end{itemize}
    \item Techniques for minimizing variance/balancing workloads (Dutot et al., 2006; Ghannadpour et al., 2013)
\end{itemize}

\subsection{Time-Constrained Route Optimization}
\begin{itemize}
    \item Handling time windows and duration constraints (Toth \& Vigo, 2002)
    \item Techniques for time-dependent weights (Gendreau et al., 2015)
    \item Applications to tourist trip planning with time budgets (Souffriau et al., 2013)
\end{itemize}

\subsection{Real-World Applications}
\begin{itemize}
    \item Implementation considerations for tourist route planning systems (Souffriau et al., 2008)
    \item Personalized itinerary systems and their algorithms (Schaller et al., 2014)
    \item Case studies of similar optimization problems in practice (Gavalas et al., 2015)
\end{itemize}

\subsection{Gap Analysis and Research Directions}
\begin{itemize}
    \item Identify how existing approaches fall short for this specific combination of constraints
    \item How methods might be adapted or combined to address the problem (Liao et al., 2017)
    \item Novel aspects requiring new algorithmic development (Nazari et al., 2018; Chou et al., 2020)
\end{itemize}

\subsection{Conclusion}
\begin{itemize}
    \item Summarize the key algorithmic approaches most relevant to this problem
    \item Identify the most promising directions based on the literature
    \item Transition to the approach taken in the current research
\end{itemize}

Full references (you'll need to format these according to your preferred citation style):

Applegate, D. L., Bixby, R. E., Chvátal, V., & Cook, W. J. (2006). The traveling salesman problem: a computational study. Princeton University Press.
Azi, N., Gendreau, M., & Potvin, J. Y. (2014). An exact algorithm for a vehicle routing problem with time windows and multiple use of vehicles. European Journal of Operational Research, 202(3), 756-763.
Cattaruzza, D., Absi, N., Feillet, D., & Vidal, T. (2016). The multi-trip vehicle routing problem with time windows and release dates. Transportation Research Part E: Logistics and Transportation Review, 92, 118-133.
Černý, V. (1985). Thermodynamical approach to the traveling salesman problem: An efficient simulation algorithm. Journal of Optimization Theory and Applications, 45(1), 41-51.
Chou, C. C., Fu, J. S., Kuo, S. P., & Li, P. H. (2020). A hybrid algorithm of simulated annealing and tabu search for the bi-objective team orienteering problem with time windows. IEEE Access, 8, 72564-72578.
Dorigo, M., & Gambardella, L. M. (1997). Ant colony system: a cooperative learning approach to the traveling salesman problem. IEEE Transactions on evolutionary computation, 1(1), 53-66.
Dutot, P. F., Laugier, A., & Bustos, A. M. (2006). Balancing a dynamic mobile workforce via constraint-based local search. In International Conference on Innovative Techniques and Applications of Artificial Intelligence (pp. 35-48).
García-Nájera, A., Bullinaria, J. A., & Gutiérrez-Andrade, M. A. (2013). A Pareto ant colony optimization algorithm for the multi-objective tourist trip design problem. In Applications of Evolutionary Computation (pp. 143-154).
Gavalas, D., Konstantopoulos, C., Mastakas, K., & Pantziou, G. (2014). Models and algorithms for the tourist trip design problem. Annals of Operations Research, 224(1), 65-86.
Gavalas, D., Konstantopoulos, C., Mastakas, K., & Pantziou, G. (2015). A variable neighborhood search approach for the tourist trip design problem. Expert Systems with Applications, 42(21), 7911-7919.
Gendreau, M., Ghiani, G., & Guerriero, E. (2015). Time-dependent routing problems: A review. Computers & Operations Research, 64, 189-197.
Ghannadpour, S. F., Noori, S., Tavakkoli-Moghaddam, R., & Ghoseiri, K. (2013). A multi-objective vehicle routing problem with soft time windows: The minimization of drivers' dissatisfaction. Transportation Research Part E: Logistics and Transportation Review, 53, 64-77.
Gunawan, A., Lau, H. C., & Vansteenwegen, P. (2016). Orienteering problem: A survey of recent variants, solution approaches and applications. European Journal of Operational Research, 255(2), 315-332.
Konak, A., Coit, D. W., & Smith, A. E. (2006). Multi-objective optimization using genetic algorithms: A tutorial. Reliability Engineering & System Safety, 91(9), 992-1007.
Laporte, G. (1992). The traveling salesman problem: An overview of exact and approximate algorithms. European Journal of Operational Research, 59(2), 231-247.
Liao, T. W., Egbelu, P. J., & Chang, P. C. (2017). Integrated manufacturing systems design with robot movement simulation and multi-objective genetic algorithms. International Journal of Production Research, 55(15), 4399-4422.
Marler, R. T., & Arora, J. S. (2010). The weighted sum method for multi-objective optimization: new insights. Structural and Multidisciplinary Optimization, 41(6), 853-862.
Nazari, M., Oroojlooy, A., Snyder, L., & Takác, M. (2018). Reinforcement learning for solving the vehicle routing problem. Advances in Neural Information Processing Systems, 31.
Necula, R., Breaban, M., & Raschip, M. (2015). Exact and heuristic approaches for the multi-visit salesman problem. In International Conference on Computational Collective Intelligence (pp. 357-368).
Potvin, J. Y. (1996). Genetic algorithms for the traveling salesman problem. Annals of Operations Research, 63(3), 337-370.
Schaller, R., Elsweiler, D., & Harvey, M. (2014). City trip planner: Personalized itinerary planning for groups of people. In Proceedings of the 22nd International Conference on User Modeling, Adaptation, and Personalization (pp. 184-191).
Souffriau, W., Vansteenwegen, P., Vertommen, J., Berghe, G. V., & Oudheusden, D. V. (2008). A personalized tourist trip design algorithm for mobile tourist guides. Applied Artificial Intelligence, 22(10), 964-985.
Souffriau, W., Vansteenwegen, P., & Oudheusden, D. V. (2013). The multi-constraint team orienteering problem with multiple time windows. Transportation Science, 47(1), 53-63.
Toth, P., & Vigo, D. (2002). The vehicle routing problem. SIAM.
Vansteenwegen, P., Souffriau, W., & Van Oudheusden, D. (2011). The orienteering problem: A survey. European Journal of Operational Research, 209(1), 1-10.