This literature review aims to explore existing research and approaches to other combinatorial optimisation problems.
There is extensive previous research on various combinatorial problems, for example, the Travelling Salesman
Problem, Vehicle Routing Problem and Tourist Trip Design Problem.
It is important to understand how these problems are similar to the one presented in this report, as well as where
those similarities end.
By gaining an understanding of the strengths and limitations of existing approaches to similar problems, we can make
better informed decisions regarding which approaches to investigate, how they may be adapted to suit our specific
constraints and how they might be implemented in practice.
While the approaches taken to these problems may not be directly applicable to our own, it is likely we can adapt their
techniques to suit the specific constraints of this problem.


The Travelling Salesman Problem (TSP) is perhaps one of the most studied optimisation problems.
This extensive research on the problem has acted as an `engine of discovery for general-purpose techniques' offering
The TSP can be described simply as: Given a set of locations and the cost of travel between them, find the shortest
route that visits each location and returns to the start\parencite[p. 1]{applegate2006traveling}.
The problem relates directly to our own in their shared goal of minimising travel time, in fact, for inputs where
$d = 1$, our problem becomes the TSP, with greater values of $d$ introducing additional complexity.

The TSP is proven to be NP-hard\parencite[p. 1096--1097{cormen2022introduction}, meaning that there
are no known algorithms capable of solving the problem in polynomial time.
Exact methods, such as brute force, branch-and-bound and dynamic programming are capable of finding optimal
solutions, they just take an impractically long time to do so.
\textcite[p. 489--530]{applegate2006traveling} discusses how the best solvers of the time had solved problems of
thousands of locations, but took many CPU years to do so.
Even with advances in computer processing, finding exact solutions appear too impractical in the context of our project.
Naturally, this leads towards the investigation of heuristic and approximation algorithms, which aim to find a near
optimal solution within a reasonable time frame.

TSP heuristics can be broadly categorised into two groups: construction heuristics and improvement heuristics.
Construction heuristics build a solution from scratch, progressively adding locations to the route, while improvement
heuristics take an existing solution and attempt to improve it\parencite[p. 242]{laporte1992traveling}.\\

\noindent
The multiple travelling salesman problem (mTSP) is an extension of the TSP, this time aiming to find a set of $m$
routes that together visit every location once, with the goal of minimising the total cost of all
routes\parencite[p. 209]{bektas2006multiple}.
This comes closer to our own problem, in which we are looking for several routes over multiple days, but without the
optimisation objective of balancing routes.\\

\noindent
The Tourist Trip Design Problem (TTDP) takes potential points of interest (PoIs) and attempts to find the most
interesting route based on a number of criteria.
TTDP shares the same motivation of tourist route planning as our own problem and often considers similar parameters,
such as travel time between PoIs and the desired time spent at each one\parencite{gavalas2014mobile}.
A large difference though, is that it does not require every location to be visited, prioritising the visitation of
certain PoIs according to user preference.
Nevertheless, due to the similarities in the problem, it is worth investigating the approaches taken to solve TTDPs.

\textcite{vansteenwegen2007mobile}



\subsection{Classical Traveling Salesman Problem (TSP)}
\begin{itemize}
    \item Definition and mathematical formulation
    \item Complexity analysis and NP-hardness
    \item Key solution approaches
    \item Relevance and limitations in relation to our specific problem
\end{itemize}

\noindent\textbf{Recommended Literature:}
\begin{itemize}
    \item Applegate, D. L., Bixby, R. E., Chvátal, V., \& Cook, W. J. (2006). \textit{The Traveling Salesman Problem: A Computational Study}. Princeton University Press.
    \item Laporte, G. (1992). ``The traveling salesman problem: An overview of exact and approximate algorithms.'' \textit{European Journal of Operational Research}, 59(2), 231-247.
    \item Lin, S., \& Kernighan, B. W. (1973). ``An effective heuristic algorithm for the traveling-salesman problem.'' \textit{Operations Research}, 21(2), 498-516.
    \item Helsgaun, K. (2000). ``An effective implementation of the Lin–Kernighan heuristic.'' \textit{European Journal of Operational Research}, 126(1), 106-130.
\end{itemize}

\subsection{Multiple Traveling Salesman Problem (mTSP)}
\begin{itemize}
    \item Extension of the TSP with multiple agents
    \item Mathematical formulation differences from TSP
    \item Application to multi-day planning scenarios
    \item Connection to our requirement of visiting the starting point $d$ times
\end{itemize}

\noindent\textbf{Recommended Literature:}
\begin{itemize}
    \item Bektas, T. (2006). ``The multiple traveling salesman problem: an overview of formulations and solution procedures.'' \textit{Omega}, 34(3), 209-219.
    \item Kara, I., \& Bektas, T. (2006). ``Integer linear programming formulations of multiple salesman problems and its variations.'' \textit{European Journal of Operational Research}, 174(3), 1449-1458.
    \item Gavish, B., \& Srikanth, K. (1986). ``An optimal solution method for large-scale multiple traveling salesmen problems.'' \textit{Operations Research}, 34(5), 698-717.
    \item Carter, A. E., \& Ragsdale, C. T. (2006). ``A new approach to solving the multiple traveling salesperson problem using genetic algorithms.'' \textit{European Journal of Operational Research}, 175(1), 246-257.
\end{itemize}

\subsection{Vehicle Routing Problem (VRP) and Variants}
\begin{itemize}
    \item Basic VRP definition and formulation
    \item Vehicle Routing Problem with Multiple Trips (VRPMT)
    \item Capacitated VRP and other variants
    \item Relevance to our balanced route planning requirement
\end{itemize}

\noindent\textbf{Recommended Literature:}
\begin{itemize}
    \item Toth, P., \& Vigo, D. (Eds.). (2002). \textit{The Vehicle Routing Problem}. SIAM Monographs on Discrete Mathematics and Applications.
    \item Cattaruzza, D., Absi, N., \& Feillet, D. (2016). ``Vehicle routing problems with multiple trips.'' \textit{4OR}, 14(3), 223-259.
    \item Brandão, J., \& Mercer, A. (1997). ``A tabu search algorithm for the multi-trip vehicle routing and scheduling problem.'' \textit{European Journal of Operational Research}, 100(1), 180-191.
    \item Olivera, A., \& Viera, O. (2007). ``Adaptive memory programming for the vehicle routing problem with multiple trips.'' \textit{Computers \& Operations Research}, 34(1), 28-47.
\end{itemize}

\subsection{Tourist Trip Design Problem (TTDP)}
\begin{itemize}
    \item Problem definition focusing on tourist-specific constraints
    \item Time-dependent considerations and point-of-interest selection
    \item Personalization aspects in tourist routing
    \item Direct applicability to our problem's tourism context
\end{itemize}

\noindent\textbf{Recommended Literature:}
\begin{itemize}
    \item Vansteenwegen, P., Souffriau, W., \& Van Oudheusden, D. (2011). ``The orienteering problem: A survey.'' \textit{European Journal of Operational Research}, 209(1), 1-10.
    \item Gavalas, D., Konstantopoulos, C., Mastakas, K., \& Pantziou, G. (2014). ``A survey on algorithmic approaches for solving tourist trip design problems.'' \textit{Journal of Heuristics}, 20(3), 291-328.
    \item Souffriau, W., Vansteenwegen, P., Vanden Berghe, G., \& Van Oudheusden, D. (2013). ``The planning of cycle trips in the province of East Flanders.'' \textit{Omega}, 41(3), 522-531.
    \item Garcia, A., Vansteenwegen, P., Arbelaitz, O., Souffriau, W., \& Linaza, M. T. (2013). ``Integrating public transportation in personalised electronic tourist guides.'' \textit{Computers \& Operations Research}, 40(3), 758-774.
\end{itemize}

\subsection{Multi-Objective Optimization in Routing Problems}
\begin{itemize}
    \item Balancing competing objectives (like total weight vs. variance)
    \item Pareto optimality concepts
    \item Solution approaches for multi-objective routing
    \item Applicability to our dual-objective function
\end{itemize}

\noindent\textbf{Recommended Literature:}
\begin{itemize}
    \item Jozefowiez, N., Semet, F., \& Talbi, E. G. (2008). ``Multi-objective vehicle routing problems.'' \textit{European Journal of Operational Research}, 189(2), 293-309.
    \item Paquete, L., \& Stützle, T. (2006). ``A study of stochastic local search algorithms for the biobjective QAP with correlated flow matrices.'' \textit{European Journal of Operational Research}, 169(3), 943-959.
    \item Laporte, G., Semet, F., Matl, P., \& Voß, S. (2018). ``Multi-objective vehicle routing problem.'' \textit{Operations Research Perspectives}, 5, 50-57.
    \item Coello, C. A. C., Lamont, G. B., \& Van Veldhuizen, D. A. (2007). \textit{Evolutionary Algorithms for Solving Multi-Objective Problems}. Springer.
\end{itemize}

\subsection{Balance-Oriented Routing Problems}
\begin{itemize}
    \item Problems focusing on workload balancing
    \item Min-max objectives in routing
    \item Variance minimization approaches
    \item Connection to our goal of minimizing variance between trips
\end{itemize}

\noindent\textbf{Recommended Literature:}
\begin{itemize}
    \item Jozefowiez, N., Semet, F., \& Talbi, E. G. (2009). ``An evolutionary algorithm for the vehicle routing problem with route balancing.'' \textit{European Journal of Operational Research}, 195(3), 761-769.
    \item Dell'Amico, M., Monaci, M., Pagani, C., \& Vigo, D. (2007). ``Heuristic approaches for the fleet size and mix vehicle routing problem with time windows.'' \textit{Transportation Science}, 41(4), 516-526.
    \item Lee, T. R., \& Ueng, J. H. (1999). ``A study of vehicle routing problems with load-balancing.'' \textit{International Journal of Physical Distribution \& Logistics Management}, 29(10), 646-657.
    \item Liu, R., Xie, X., Augusto, V., \& Rodriguez, C. (2013). ``Heuristic algorithms for a vehicle routing problem with simultaneous delivery and pickup and time windows in home health care.'' \textit{European Journal of Operational Research}, 230(3), 475-486.
\end{itemize}

\subsection{Synthesis and Research Gaps}
\begin{itemize}
    \item Comparison of problem characteristics across reviewed literature
    \item Key methodological approaches applicable to our problem
    \item Identification of research gaps in addressing our specific problem constraints
    \item Potential directions for adaptation of existing methodologies
\end{itemize}

\noindent\textbf{Recommended Literature:}
\begin{itemize}
    \item Laporte, G. (2009). ``Fifty years of vehicle routing.'' \textit{Transportation Science}, 43(4), 408-416.
    \item Cordeau, J. F., Gendreau, M., Laporte, G., Potvin, J. Y., \& Semet, F. (2002). ``A guide to vehicle routing heuristics.'' \textit{Journal of the Operational Research Society}, 53(5), 512-522.
    \item Eksioglu, B., Vural, A. V., \& Reisman, A. (2009). ``The vehicle routing problem: A taxonomic review.'' \textit{Computers \& Industrial Engineering}, 57(4), 1472-1483.
    \item Vidal, T., Crainic, T. G., Gendreau, M., \& Prins, C. (2013). ``Heuristics for multi-attribute vehicle routing problems: A survey and synthesis.'' \textit{European Journal of Operational Research}, 231(1), 1-21.
\end{itemize}

\subsection{Conclusion}
\begin{itemize}
    \item Summary of most relevant approaches
    \item Recommendation for methodological direction
    \item Justification for selected approach based on literature findings
\end{itemize}