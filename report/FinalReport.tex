% Preamble
\documentclass[a4paper,12pt]{article}

% Prevents line breaks mid-word
\tolerance=1
\emergencystretch=\maxdimen
\hyphenpenalty=10000
\hbadness=10000

% Packages
\usepackage{amsmath}
\usepackage{graphicx}
\usepackage{blindtext}
\usepackage{titlesec}
\graphicspath{{images/}}
\usepackage{todonotes}

 % Set inline as default for todos
\usepackage{regexpatch}
\usepackage{amsfonts}
\makeatletter
\xpatchcmd{\@todo}{\setkeys{todonotes}{#1}}{\setkeys{todonotes}{inline,#1}}{}{}
\makeatother

% Document
\begin{document}
    \begin{titlepage}
        \centering
        \includegraphics[width = 1\textwidth]{UoB_Logo}\\[10ex]
        \Huge{A Comparison of Approaches to Combinatorial Optimization for Touristic Route Planning}\\[14ex]
        \LARGE{Jacob Luck}\\
        \large{2338114}\\[2ex]
        \Large{BSc Computer Science with a Year in Industry}\\
        \Large{Project Supervisor - Leandro Minku}\\[2ex]
        \large{April 2025}
    \end{titlepage}

    \tableofcontents

    \pagebreak

    \addcontentsline{toc}{section}{Abstract}
    \section*{Abstract}
    \todo{Write abstract}

    \section{Introduction}\label{sec:introduction}
    \todo{Write introduction}

    \pagebreak
    
    \section{Problem Formulation}\label{sec:problem-formulation}
    \subsection{Problem Description}\label{subsec:problem-description}
    Given a positive integer $d$ and a graph $G = (V, E)$, where $V$ is set of locations including a designated
    central node $C$ and $E$ is a set of weighted edges linking every location to every other location, as well as a
    duration function $D(v)$ assigning a duration to each node $v \in V$, find a route that:
    \begin{enumerate}
        \item Visits all nodes $V \setminus C$ once.
        \item Starts and finishes at $C$, having visited it $d$ times, without ever visiting consecutively.
        \item Minimizes both the cumulative edge weights in the route and the variance in cumulative weight between
        each visit to $C$.
    \end{enumerate}\\
    \subsection{Inputs and Outputs}\label{subsec:inputs-and-outputs}
    Inputs:
    \begin{itemize}
        \item $d \in \mathbb{Z}^+$: The number of times $C$ should be visited in a route.
        Contextually, $d$ represents the number of days a tourist will spend on their trip.
        \item $G = (V, E)$: A pair comprising:
        \begin{itemize}
            \item[\textbullet] $V$: A set of nodes $v \in V$.
            \begin{itemize}
                \item[\textbullet] $v \in V$: A pair of latitude and longitude coordinates\footnote{While the
                coordinates of our locations are included in $V$, they are not directly tied to the weight of our
                edges $E$, which are based on travel time and not distance.}.
                Contextually, $v$ represents a location the tourist would like to visit in their trip.
            \end{itemize}
            \item[\textbullet] $E$: A set of edges $(a, b) \in E$ that connects every node to every other node,
            bidirectionally.
            \begin{itemize}
                \item[\textbullet] $(a, b) \in E$: A directional edge linking $a$ and $b$ with a weight indicating
                travel time.
                Contextually, $(a, b)$ represents how a tourist will get from $a$ to $b$.
            \end{itemize}
        \end{itemize}
        \item $C \in V$: Central node which should be visited $d$ times.
        Contextually, $C$ represents where the tourist is staying and will return to at the end of each day.
    \end{itemize}

    \pagebreak

    \section{Literature Review}\label{sec:literature-review}
    \todo{Write literature review}

    \pagebreak

    \section{Algorithms Investigated}\label{sec:algorithms-investigated}
    \todo{Paragraph describing different types of algorithm used (Routing, Cluster and Routing, Evolutionary, etc.)}
    \subsection{Brute Force}\label{subsec:brute-force}
    \todo{Write brute force explanation}

    \subsection{Clustering}\label{subsec:clustering}
    \todo{Write clustering explanation}
    Once divided into clusters, each cluster can be solved using our brute force method (with \(d = 1\)) or with
    traditional TSP algorithms.
    TSP algorithms implemented in this project include:
    \subsection{Greedy}\label{subsec:greedy}
    \todo{Write greedy explanation}

    \pagebreak
    
    \section{Evaluation and Comparison}\label{sec:evaluation-and-comparison}
    \todo{Write paragraph about experiment process. Comparison based on computation time and route evaluation.
    Describe how route is evaluated. Describe data being tested on.}

    \pagebreak

    \section{Conclusion}\label{sec:conclusion}
    \todo{Write conclusion}

    \pagebreak

    \section{Bibliography}\label{sec:bibliography}
    \todo{Fill in bibliography}
\end{document}