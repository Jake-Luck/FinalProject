\todo{There should be a link either here or at end of literature which forms the basic for different methods (clustering, routing, trip generation).}
\todo{Paragraph describing different types of algorithm used (Routing then cluster, Cluster then Routing, Genetic, etc.)}
\todo{Remember to justify the choice of algorithms. You may also need to explain how to adopt these algorithms in your work. A figure showing the ralationship between different components of your work may also help.}

\subsection{Objective Function}\label{subsec:algorithms-objective-function}
Before describing the algorithms investigated, we will quickly cover the implementation of our objective function.
As described in the \hyperref[sec:problem-formulation]{Problem Formulation}, our objective function aims to
calculate a cost for a given route.
This function is needed such that we can compare our approaches and algorithms based on the quality of their results.
Figure~\ref{fig:Algorithm.evaluate_route} shows how this was implemented in our project.
\todo{Emphasise how graph is weighted using time instead of distance and why these don't directly correlate}
\begin{figure}[H]
    \centering
    \includegraphics[width = \textwidth]{Algorithm.evaluate_route}
    \caption{Code from Algorithm.evaluate\_route in algorithms\textbackslash algorithm.py}
    \label{fig:Algorithm.evaluate_route}
\end{figure}

\noindent
All the code used in the project is available to find at: \url{https://git.cs.bham.ac.uk/projects-2024-25/jhl114}.
Code snippets shown in this report, such as figure~\ref{Algorithm.evaluate_route}, are captioned
detailing where in the git repository their code can be found.

\subsection{Clustering}\label{subsec:clustering}
Clustering as a concept can be described as `the unsupervised classification of patterns (observation, data items, or
feature vectors) into groups (clusters)'\todo{cite Data Clustering: A review, A.K. Jain}.
In our problem, clustering will be used to group locations together to form an itinerary for each day of the trip.
These clusters, or days, will then be used as an input for some routing algorithm, which will try and find an
optimal route for each day.
These routes can then be combined to form a complete trip.
The goal of our clustering algorithms is to find a set of clusters that, when combined with some routing algorithm,
will produce a trip with minimal cost.
The code in figure~\ref{fig:Clustering.find_route_from_clusters} shows how we can use a set of clusters alongside our
graph and duration inputs to find a trip.
\begin{figure}[H]
    \centering
    \includegraphics[width = \textwidth]{Clustering.find_route_from_clusters}
    \caption{Code from Clustering.find\_route\_from\_clusters in algorithms\textbackslash clustering.py}
    \label{fig:Clustering.find_route_from_clusters}
\end{figure}

\noindent
The clustering algorithms implemented in this project are: K-Means, Genetic Clustering and Genetic Centroid-based
Clustering.

\subsubsection{K-Means}\label{subsubsec:k-means}
K-Means is an iterative clustering algorithm that defines its clusters using a set of centroids (means) which are
given a location in the input space, a given location is assigned to the cluster of the `closest' centroid.
The algorithm starts by initialising random centroids and iteratively improving the clustering from there, continuing
until the algorithm converges (on a local optimum) or an iteration limit is reached.
While the algorithm may not find the best solution, it is rather quick, with a time complexity of $O(m n k i)$,
where $m$ is the number of locations, $n$ is the dimensionality of the input, $k$ is the number of clusters,
and $i$ is the number of iterations\todo{cite Algorithm AS 136: A K-Means Clustering Algorithm}.
Our inputs will always contain only two dimensions, and we will be setting a maximum number of iterations, this
makes both $n$ and $i$ constants allowing us to simplify the time complexity to $O(mk)$.\\

\noindent
In our implementation of K-Means we will initialise our centroids by randomly selecting unique locations from our
input, and placing our centroids at their coordinates.
A different approach was considered, which involved initialising the centroids with random coordinates in a similar
area to the input, however this had the potential to create clusters with zero locations assigned to them resulting
in invalid trips.
By starting with locations from the input, we can be sure that all clusters have at least one location assigned to them.
We will be using the coordinates of our locations to calculate the Euclidean distances between locations and centroids,
these locations will be assigned to the cluster of the closest centroid.
Figure~\ref{fig:Clustering._assign_nodes_to_centroid} shows how this was accomplished.
\begin{figure}[H]
    \centering
    \includegraphics[width = \textwidth]{Clustering._assign_nodes_to_centroid}
    \caption{Code from Clustering.\_assign\_nodes\_to\_centroid in algorithms\textbackslash clustering.py}
    \label{fig:Clustering._assign_nodes_to_centroid}
\end{figure}

\noindent
After assignment, the centroids are recalculated such that their coordinates are the average of all locations
assigned to their cluster.
This is done by iterating through each cluster and calculating the mean of all locations assigned to it.
Our implementation is shown in figure~\ref{fig:KMeans._compute_means}.
\begin{figure}[H]
    \centering
    \includegraphics[width = \textwidth]{KMeans._compute_means}
    \caption{Code from KMeans.\_compute\_means in algorithms\textbackslash clustering.py}
    \label{fig:KMeans._compute_means}
\end{figure}

\noindent
These steps of cluster assignment and centroid recalculation are repeated until either a maximum allowed number of
iterations is reached, or until the algorithm converges on an optimum solution.
Our convergence criterion is that the centroids stop changing between iterations, i.e., the centroids are the same
as the previous iteration.
Our python implementation of this is shown in figure~\ref{fig:KMeans.find_clusters}.
\begin{figure}[H]
    \centering
    \includegraphics[width = \textwidth]{KMeans.find_clusters}
    \caption{Code from KMeans.find\_clusters in algorithms\textbackslash clustering.py}
    \label{fig:KMeans.find_clusters}
\end{figure}

\noindent
Figure~\ref{fig:KMeans_London_Step1} shows an example of the iterations of a K-Means algorithm run on an input with 25
points of interest around London to be visited over seven days.
\begin{figure}[H]
    \ContinuedFloat*
    \includegraphics[width = \textwidth]{KMeans_London_Step1}
    \caption{K-Means example Step 1, Initial centroid positions and cluster assignments.}
    \label{fig:KMeans_London_Step1}
\end{figure}
\begin{figure}[H]
    \ContinuedFloat
    \includegraphics[width = \textwidth]{KMeans_London_Step2}
    \caption{K-Means example Step2, Centroids have been updated, locations are reassigned to their closest centroid.}
    \label{fig:KMeans_London_Step2}
\end{figure}
\begin{figure}[H]
    \ContinuedFloat
    \includegraphics[width = \textwidth]{KMeans_London_Step3}
    \caption{K-Means example Step3, Centroids updated again and locations updated.}
    \label{fig:KMeans_London_Step3}
\end{figure}
\begin{figure}[H]
    \ContinuedFloat
    \includegraphics[width = \textwidth]{KMeans_London_Step4}
    \caption{K-Means example Step4, Centroids have updated and no locations have changed cluster, a solution has been found.}
    \label{fig:KMeans_London_Step4}
\end{figure}

\noindent
It is worth noting that K-Means only forms these clusters based on Euclidean distances, grouping together locations
that are close geographically.
However, as formally described in the \hyperref[subsec:objective-function]{Objective Function} section of the
\hyperref[sec:problem-formulation]{Problem Formulation}, a good trip will minimise both the route length of the trip
and the variance between time spent each day.
K-Means does not aim to optimise for the variance between days, it doesn't even consider the time spent at each
location.
Furthermore, while each cluster might be optimised for distance, how close two locations are may not reflect the travel
time between them.
While K-Means does not intentionally optimise for variance between days or consider travel time between locations, it
was still chosen for this project out of curiosity as to how effective a heuristic it might provide.
Hopefully it will offer a simple and computationally efficient baseline for comparison with more complex
algorithms.

\subsubsection{Genetic Clustering}
\todo{Explain genetic clustering}
Genetic clustering applies genetic algorithms to attempt and find the best assignment of locations to clusters.
Genetic algorithms are a type of evolutionary algorithm that aims to mimic biological evolution to find an optimal
solution.
They involve creating a population of potential solutions (individuals) and iteratively improving the population
through selection (keeping the best individuals in the population, akin to natural selection), crossover (combining
individuals to create offspring, akin to sexual reproduction), and mutation (randomly altering the genomes
of individuals in the population, akin to biological mutation).

For us to perform selection, and find the best solutions in a population, we need to assign some fitness to each
individual.
To do this we will combine the cluster assignments with a chosen routing algorithm, and then apply our cost function
to the route found.
This cost will be used to rank our population, helping us find clusters that can produce a good trip.

The performance of Genetic algorithms is highly dependent on its hyperparameters: mutation rate, determining how
common mutation is; crossover rate, determining how often offspring are created via crossover, as opposed to new
additions to the population; population size, determining how many individuals there are per generation; number of
generations, determining how many generations will be evolved to reach a solution; crossover method used,
determining how crossover is performed to create offspring; and in our case, the routing algorithm used, which may
impact how clusters are used to form routes.
These hyperparameters impact both the runtime of the algorithm and the exploration of the search space, indirectly
impacting the quality of the solution.
Genetic clustering has a time complexity of $O(gpr)$, with $g$ being the number of generations, $p$ being the
population size and $r$ being the time complexity of the chosen routing algorithm.\\

\noindent
For this genetic clustering, an individual is represented by a genome, which will provide a mapping that assigns each
location to a cluster.
Figure~\ref{fig:barcelona-genome-example} shows an example of this.
\begin{figure}[H]
    \centering
    Genome: [0, 0, 0, 1, 1, 1, 2, 2, 2]\\
    \includegraphics[width = \textwidth]{Barcelona_Genome_Example}
    \caption{Example of how an individual's genome corresponds to cluster assignments.}
    \label{fig:barcelona-genome-example}
\end{figure}

\noindent
These genomes are our target for performing crossover and mutations.
We begin our evolution process by randomly generating an initial population of individuals.
From there, we repeat the following steps until we reach a maximum number of generations:
\begin{enumerate}
    \item Evaluate the fitness of each individual in the population.
    \item Select the best individuals from the population, these will be carried over into the next generation, as
    well as be used to create offspring.
    \item Generate new population using crossover and mutation.
\end{enumerate}

\noindent
As previously discussed, we will evaluate the fitness of each individual by applying a routing algorithms to the
clusters defined by the genome, and then applying our cost function to the resulting route.
Figure~\ref{fig:GeneticClustering._evaluate_population} shows this route finding and evaluation.
\begin{figure}
    \centering
    \includegraphics[width = \textwidth]{GeneticClustering._evaluate_population}
    \caption{Code from GeneticClustering.\_evaluate\_population in algorithms\textbackslash clustering.py}
    \label{fig:GeneticClustering._evaluate_population}
\end{figure}

\noindent
The methods called in figure~\ref{fig:GeneticClustering._evaluate_population} are those previously shown in
figure~\ref{fig:Clustering.find_route_from_clusters} (find_route_from_clusters) and
figure~\ref{fig:Algorithm.evaluate_route} (evaluate_route).
Once the population has been evaluated, the best two individuals are chosen as parents, as figure~\ref{fig:GeneticClustering.find_clusters.select_parents}
shows.
\begin{figure}[H]
    \centering
    \includegraphics[width = \textwidth]{GeneticClustering.find_clusters.select_parents}
    \caption{Code from GeneticClustering.find\_clusters in algorithms\textbackslash clustering.py}
    \label{fig:GeneticClustering.find_clusters.select_parents}
\end{figure}

\noindent
Excluding the two parents, who will be copied over, the next generation will be created via crossover and mutation,
or through random generation.
We include some randomly generated individuals in an effort to increase genetic diversity and exploration of the
search space.
Figure ~\ref{fig:GeneticClustering.find_clusters.crossover} shows how this is decided in our implementation.
For each individual a random number is generated between 0 and 1, if this number is lower than our crossover rate
the individual is created via crossover, otherwise it will be randomly generated.
\begin{figure}[H]
    \centering
    \includegraphics[width = \textwidth]{GeneticClustering.find_clusters.crossover}
    \caption{Code from GeneticClustering.find\_clusters in algorithms\textbackslash clustering.py}
    \label{fig:GeneticClustering.find_clusters.crossover}
\end{figure}

\noindent
For our implementation of crossover, we will be performing a uniform crossover, using crossover masks.
A crossover mask is a bit array the same length as the genome, with the parity of each bit indicating which parent
to choose from for the corresponding bit in the created genome.
In a uniform crossover mask, each bit has an 50\% chance of being a 0 or 1, meaning that each bit in the offspring
genome has an equal chance of being from either parent
\todo{Cite Uniform Crossover in Genetic Algorithms, Giblert Syswerda.}.
If the two parents, being the best individuals in a population, agree on a bit it then it would appear likely it's a
good choice.
With our implementation of crossover, the offspring will always copy over the bits that parents agree on.
Figure~\ref{fig:Crossover_Mask_Example} shows an example of how a crossover mask can be used to create a child from
two parents, the example uses a hypothetical input including six locations and three clusters.
\begin{figure}[H]
    \centering
    \includegraphics[width = \textwidth]{Crossover_Mask_Example}
    \caption{Example of using a crossover mask to create a child from two parents.}
    \label{fig:Crossover_Mask_Example}
\end{figure}

\noindent
There is however, one slight issue with applying this method to our problem.
In the example shown in figure~\ref{fig:Crossover_Mask_Example}, parent 1's 0th cluster only includes the first
location, similarly, parent 2's 2nd cluster also only includes the first location.
Both parents are forming a cluster using these locations, however due to them being labelled differently, the
offspring produced did not continue this clustering.
To solve this problem a genome's clusters will be relabelled in order of their appearance in the genome, ensuring
consistency between parents.
Figure~\ref{fig:Crossover_Mask_Relabelling_Example} shows an example of this applied to the parents in figure~\ref{fig:Crossover_Mask_Example},
and figure~\ref{fig:GeneticClustering._relabel_individuals_clusters} shows how this is accomplished in our python
implementation.
\begin{figure}[H]
    \centering
    \includegraphics[width = \textwidth]{Crossover_Mask_Relabelling_Example}
    \caption{Example of relabelling a parent's clusters and the resultant offspring}
    \label{fig:Crossover_Mask_Relabelling_Example}
\end{figure}
\begin{figure}[H]
    \centering
    \includegraphics[width = \textwidth]{GeneticClustering._relabel_individuals_clusters}
    \caption{Code from GeneticClustering.\_relabel\_individuals\_clusters in algorithms\textbackslash clusterin.py}
    \label{fig:GeneticClustering._relabel_individuals_clusters}
\end{figure}

\noindent
After this relabelling, crossover can be performed without issue.
Figure~\ref{fig:GeneticClustering._crossover} shows our python implementation of this.
\begin{figure}[H]
    \centering
    \includegraphics[width = \textwidth]{GeneticClustering._crossover}
    \caption{Code from GeneticClustering.\_crossover in algorithms\textbackslash clustering.py}
    \label{fig:GeneticClustering._crossover}
\end{figure}

After crossover is complete, we randomly mutate the created offspring in the hopes of increasing genetic diversity
and escaping local optima.
We mutate our genome by iterating through each gene and randomly deciding if it will mutate or not, the likeliness
of mutation is decided by our mutation rate.
If a gene is chosen to mutate, it will assign itself to a random cluster.
Figure~\ref{fig:GeneticClustering.find_clusters.mutation} shows how this is implemented in python.
\begin{figure}[H]
    \centering
    \includegraphics[width = \textwidth]{GeneticClustering.find_clusters.mutation}
    \caption{Code from GeneticClustering.find\_clusters in algorithms\textbackslash clustering.py}
    \label{fig:GeneticClustering.find_clusters.mutation}
\end{figure}

\noindent
After the new population has been created, we restart the process for the next generation, repeating these steps
until a maximum number of generations is reached.
By time evolution is complete the population will hopefully have converged on a good solution.
\todo{Insert figure and reference it}

\noindent
\todo{Discuss how clusters may look unusual due to non-Euclidean distances.}
\todo{Discuss why chosen, no clue what to put here, I just did it bc it's cool.}

\subsubsection{Centroid-based Genetic Clustering}
\todo{Explain genetic centroid-based clustering and how it differs from general clustering.}
Inspired by K-Means, centroid-based genetic clustering uses a genetic algorithm to find the best set of centroids
to cluster the data.
The same process of evolution is used, as described previously, except this time the genome will specify the
centroids to use for clustering.
With a different genome structure, we will have to reconsider our crossover and mutation methods.
Centroid based genetic clustering keeps the same time complexity as genetic clustering, $O(gpr)$, with $g$ being the
number of generations, $p$ being the population size and $r$ being the time complexity of the chosen routing.\\

\noindent
Our new genome will be a list of coordinate pairs, representing the latitude and longitude of each centroid.
Figure~\ref{fig:GeneticCentroids_Greenwich_Example} shows an example of how locations are clustered using our centroid-based genome.
\begin{figure}[H]
    \centering
    Genome: $\begin{bmatrix}0.00 & 0.01 & -0.01 & 0.00 & 0.00\\51.477 & 51.477 & 51.477 & 51.482 & 51.472\end{bmatrix}$
    \includegraphics[width = \textwidth]{GeneticCentroids_Greenwich_Example}
    \caption{Example of how an individual's genome corresponds to cluster centroids.}
    \label{fig:GeneticCentroids_Greenwich_Example}
\end{figure}

\noindent
Once again, we will iteratively evolve our population through the process of selection, crossover and mutation.
Our selection procedure is largely the same in that we evaluate each individual by generating a trip with the help
of its genome and calculating the associated cost of these trips.
The only difference is that, because our genome is no longer a direct assignment of clusters, we first need to
assign each location to its nearest centroid.
This is shown in figure~\ref{fig:GeneticCentroids._evaluate_population}.
\begin{figure}
    \centering
    \includegraphics[width = \textwidth]{GeneticCentroids._evaluate_population}
    \caption{From GeneticCentroidClustering.\_evaluate\_population in algorithms\textbackslash clustering.py}
    \label{fig:GeneticCentroids._evaluate_population}
\end{figure}

\noindent


\subsection{Routing}\label{subsec:routing}
\todo{Explain purpose of routing/goal of algorithms.}
\subsubsection{Brute Force}\label{subsubsec:brute-force-routing}
\todo{Write brute force explanation}
The brute force algorithm is an exhaustive algorithm that tries every possible route to find one with the least cost.
By checking every route it is guaranteed to find the optimal route, however, its computational cost becomes
impractical as input size grows, having a time complexity of $O(n!)$.\todo{Maybe cite time complexity of brute force?}
Considering we will be comparing algorithms based on their speed and the quality of their results, brute force is a
useful benchmark, providing a lower bound for speed and an upper bound for quality.\\
\\
In our brute force implementation, where n is the number of locations in the route, we will be generating all $n-1!$
permutations of the set $\{1, 2, \mathellipsis, n-1\}$, with each permutation representing the order of locations
visited in a route.
Each route will be evaluated according to our optimisation function, and the route with the lowest cost will be
returned.
We only need to consider $n-1!$ permutations because all our routes will start and end at the same location.\\


\subsubsection{Greedy Routing}\label{subsubsec:greedy-routing}
\todo{Explain greedy routing algorithm}
Greedy routing
\subsubsection{Gift Wrapping}\label{subsubsec:gift-wrapping}
\todo{Explain gift wrapping algorithm}
\todo{Something like: "Once gift wrapping has found a convex hull, a greedy insertion algorithm is used to find the optimal route within the convex hull."}
\subsubsection{Genetic Routing}
\todo{Explain genetic routing}

\subsection{Route Insertion}\label{subsec:route-insertion}
\todo{Explain route insertion, how it is used in route planning and the goal of our algorithms.}
\subsubsection{Brute Force}\label{subsubsec:brute-force-route-insertion}
\todo{Explain how brute force algorithm can be modified for route insertion.}
\subsubsection{Greedy Insertion}\label{subsubsec:greedy-insertion}
\todo{Explain how greedy algorithm can be modified for route insertion.}

\subsection{Trip Generation}\label{subsec:trip-generation}
\todo{Explain trip generation, how it is used in route planning and the goal of our algorithms.}
\subsubsection{Brute Force}\label{subsubsec:brute-force-trip-generation}
\todo{Explain how brute force algorithm can be modified for trip generation.}
\subsubsection{Genetic Trip Generation}
\todo{Explain genetic trip generation}