\todo{There should be a link either here or at end of literature which forms the basic for different methods (clustering, routing, trip generation).}
\todo{Paragraph describing different types of algorithm used (Routing then cluster, Cluster then Routing, Genetic, etc.)}
\todo{Remember to justify the choice of algorithms. You may also need to explain how to adopt these algorithms in your work. A figure showing the ralationship between different components of your work may also help.}

\subsection{Clustering}\label{subsec:clustering}
\todo{Explain purpose of clustering, how it is used in route planning and the goal of our algorithms.}
\subsubsection{K-Means}\label{subsubsec:k-means}
\todo{Explain k-means algorithm}
\subsubsection{Genetic Clustering}
\todo{Explain genetic clustering}
\subsubsection{Genetic Centroid-based Clustering}
\todo{Explain genetic centroid-based clustering and how it differs from general clustering.}

\subsection{Routing}\label{subsec:routing}
\todo{Explain purpose of routing/goal of algorithms.}
\subsubsection{Brute Force}\label{subsubsec:brute-force-routing}
\todo{Write brute force explanation}
The brute force algorithm is a simple one that simply tries every possible route and returns the best one.
It is guaranteed to find the optimal route, however, its computational cost becomes impractical as input size grows,
with a time complexity of $O(n!)$.
Considering we will be comparing algorithms based on their speed and the quality of their results, brute force is a
useful benchmark, providing a lower bound for speed and an upper bound for quality.\\\\
In our brute force implementation, where n is the number of locations in the route, we will be generating all $n-1!$
permutations of the set $\{1, 2, \mathellipsis, n-1\}$, with each permutation representing the order of locations
visited in a route.
Each route will be evaluated according to our optimisation function, and the route with the lowest cost will be
returned.
We only need to consider $n-1!$ permutations because all our routes will start and end at the same location.\\


\subsubsection{Greedy Routing}\label{subsubsec:greedy-routing}
\todo{Explain greedy routing algorithm}
\subsubsection{Gift Wrapping}\label{subsubsec:gift-wrapping}
\todo{Explain gift wrapping algorithm}
\todo{Something like: "Once gift wrapping has found a convex hull, a greedy insertion algorithm is used to find the optimal route within the convex hull."}
\subsubsection{Genetic Routing}
\todo{Explain genetic routing}

\subsection{Route Insertion}\label{subsec:route-insertion}
\todo{Explain route insertion, how it is used in route planning and the goal of our algorithms.}
\subsubsection{Brute Force}\label{subsubsec:brute-force-route-insertion}
\todo{Explain how brute force algorithm can be modified for route insertion.}
\subsubsection{Greedy Insertion}\label{subsubsec:greedy-insertion}
\todo{Explain how greedy algorithm can be modified for route insertion.}

\subsection{Trip Generation}\label{subsec:trip-generation}
\todo{Explain trip generation, how it is used in route planning and the goal of our algorithms.}
\subsubsection{Brute Force}\label{subsubsec:brute-force-trip-generation}
\todo{Explain how brute force algorithm can be modified for trip generation.}
\subsubsection{Genetic Trip Generation}
\todo{Explain genetic trip generation}