% Preamble
\documentclass[a4paper,12pt]{article}

% Prevents line breaks mid-word
\tolerance=1
\emergencystretch=\maxdimen
\hyphenpenalty=10000
\hbadness=10000

% Packages
\usepackage{amsfonts}
\usepackage{amsmath}
\usepackage{blindtext}
\usepackage{float}
\usepackage[margin=1in]{geometry}
\usepackage[hidelinks]{hyperref}
\usepackage{longtable}
\usepackage{titlesec}
\usepackage{todonotes}
\usepackage{setspace}
\usepackage{subcaption}


\usepackage{caption}
\DeclareCaptionLabelFormat{cont}{#1~#2\alph{ContinuedFloat}}
\captionsetup{justification=centering, margin=0.5cm}
\captionsetup[ContinuedFloat]{labelformat=cont}

\usepackage{graphicx}
\graphicspath{{images/}}
% Uncomment command to hide all pictures - used for faster compilation
\renewcommand{\includegraphics}[2][]{\fbox{#2}}


\usepackage[style=authoryear,backend=biber]{biblatex}
\addbibresource{bibliography.bib}

 % Set inline as default for todos
\usepackage{regexpatch}
\usepackage{array}
\makeatletter
\xpatchcmd{\@todo}{\setkeys{todonotes}{#1}}{\setkeys{todonotes}{inline,#1}}{}{}
\makeatother

% Document
\begin{document}
    \begin{titlepage}
        \centering
        \vspace*{8ex}
        \includegraphics[width = \textwidth]{UoB_Logo}\\[10ex]
        \Huge{A Comparison of Approaches to Combinatorial Optimisation for Multi-Day Route Planning}\\[3ex]
        \LARGE{Jacob Luck}\\
        \large{2338114}\\[18ex]
        \Large{B.Sc. Computer Science with a Year in Industry}\\
        \Large{Project Supervisor - Leandro Minku}\\[6ex]
        \large{April 2025}\\
        \large{9814 Words}
    \end{titlepage}

    \doublespacing

    \section*{Acknowledgements}\label{sec:acknowledgements}
    I would like to thank openrouteservice and OpenStreetMap contributors, without whose efforts this project may
    not have been possible.\\

    \noindent
    I would also like to thank my project supervisor, Leandro Minku, for his guidance and support throughout the
    development of this project and report.\\

    \pagebreak

    \section*{Abstract}\label{sec:abstract}
    The Multi-Day Trip Planning problem presented in this report is a combinatorial optimisation problem that aims
    to plan a trip across multiple days, visiting a specified set of locations.
    The problem is optimised according to the amount of time spent travelling in the trip, and the balance of time
    spent across each day.
    An effective solution to this problem would allow a user to easily plan trips that efficiently visit the locations
    they wish to see.

    In this report a number of different approaches to solving this problem are investigated, including algorithms
    shown to be effective in similar optimisation problems.
    These approaches are evaluated against each other to try and determine the most practical solutions to the problem.
    By the end of this report it is shown that a combination of genetic clustering algorithms with approximate and
    exact routing methods seem to be the most promising approaches.

    \pagebreak

    \tableofcontents

    \pagebreak

    \section{Introduction}\label{sec:introduction}
    The Multi-Day Trip Planning problem investigated in this project can be simply described like so: given several
locations, and a number of days over which the trip will take place, find a route that visits all locations while
minimising the time spent travelling and distributing time spent evenly across each day.
The motivation behind this problem is centred around tourism and travel, where a user may want to visit several
points of interest in a city or region, and would like to find an efficient route for doing so across a number of days.

This project aims to investigate different potential approaches to solving this problem, and to evaluate the performance
of these approaches against each other.
The structure of this report will be as follows:
\begin{itemize}
    \item In section~\ref{sec:problem-formulation} a mathematical formulation of the Multi-Day Trip Planning problem
    will be introduced.
    \item Section~\ref{sec:literature-review} will examine and analyse existing research into problems similar to
    that covered in this project, and the approaches taken to solve them.
    \item Section~\ref{sec:algorithms-investigated} discusses the algorithms that have been investigated and
    implemented in this project.
    \item Section~\ref{sec:evaluation-and-results} will present the methodology used to evaluate these
    algorithms, and present the results of the experiments ran.
    \item Finally, section~\ref{sec:conclusion-and-future-work} will conclude this report, reviewing the project and
    any future work that could be done.
\end{itemize}\\

\noindent
All the code used in the project is available at: \url{https://git.cs.bham.ac.uk/projects-2024-25/jhl114}.
Code snippets shown in this report are captioned detailing where in the gitlab repository their code can be found.

    \pagebreak

    \section{Problem Formulation}\label{sec:problem-formulation}
    \subsection{Problem Description}\label{subsec:problem-description}
Given a positive integer: $d$, the number of days in the trip; a graph $G = (V, E)$, where $V$ is set of locations
including a designated starting point $s$, and $E$ is a set of weighted edges linking every location to every other
location; and a duration function $D(v), v \in V$, which assigns a visit duration to each node $v$, find a route that:
\begin{enumerate}
    \item Visits all nodes $V \setminus s$ once.
    \item Starts and finishes at $s$, having visited it $d$ times, without ever visiting consecutively.
    \item Minimises both the cumulative edge weights in the route and the variance in cumulative weight between
    each visit to $s$.
\end{enumerate}

\subsection{Inputs, Outputs and Design Variables}\label{subsec:inputs-and-outputs}
Inputs:
\begin{itemize}
    \item $d$: The number of times $s$ should be visited in a route.
    Contextually, $d$ represents the number of days a tourist will spend on their trip. $d \in \mathbb{Z}, d > 0$
    \item $D(v)$: A duration functions that assigns a duration to each location.
    Contextually this represents the amount of time a tourist would like to spend visiting each point of location.
    \item $G = (V, E)$: A pair comprising:
    \begin{itemize}
        \item[\textbullet] $V$: A set of nodes representing locations the tourist would like to visit.
        $v \in V, v = (x, y, t)$, a triple comprising:
        \begin{itemize}
            \item[\textbullet]$x$: Longitude, indicating the location's geographic east-west position on the earth
            $x \in \mathbb{Q}, -180 \leq x \leq 180$.
            \item[\textbullet]$y$: Latitude, indicating the location's geographic north-south position on the earth
            $y \in \mathbb{Q}, -90 \leq y \leq 90$.
        \end{itemize}
        \item[\textbullet] $E$: A set of edges $e \in E$ that connects every node to every other node,
        bidirectionally. $e = (v_1, v_2, w)$, a triple comprising:
        \begin{itemize}
            \item[\textbullet]$v_1$: A location representing the origin of the edge.\\
            $v_1 \in V$.
            \item[\textbullet]$v_2$: A location representing the destination of the edge.\\
            $v_2 \in V$.
            \item[\textbullet]$w$: A weight indicating the sum of the time it takes to travel from $v_1$ to $v_2$
            and the time the tourist wishes to spent at $v_2$.\\
            $w \in Z, w > 0$.
        \end{itemize}
    \end{itemize}
    \item $s$: Starting point that should be visited $d$ times.
    Contextually, $s$ represents where the tourist is staying and will return to at the end of each day.\\
    $s \in V$.
\end{itemize}
Outputs:
\begin{itemize}
    \item $R$: A valid route satisfying all constraints, represented as an ordered sequence of locations.\\
    $R = [r_1, r_2, \dots, r_n], r_i \in V$.
\end{itemize}

\subsection{Objective Function}\label{subsec:objective-function}
As previously mentioned in the \hyperref[subsec:problem-description]{Problem Description}, the goal of this problem is
to find a route that minimises the cumulative weight and the variance in route weight between each visit to $s$.
To accomplish this the following cost function is applied to each route:
\begin{equation}
    Cost(R) = W/d \times (1 + \sigma^2)\label{eq:cost}
\end{equation}
Where $W$ is the sum of the weights of all edges traversed in the route and $\sigma^2$ is the variance of the
sum of weights between each visit to $s$:
\begin{equation}
    W = \sum_{i=0}^{n-1} w(r_i, r_{i+1}) + D(v_i), r_i \in R, v_i \in V\label{eq:weight}
\end{equation}
Where $w(r_i, r_{i+1})$ is the weight of the edge between $r_i$ and $r_{i+1}$.
\begin{equation}
    \sigma^2 = \frac{\sum_{i=0}^{d}(x_i-\mu)}{d}, x_i \in X\label{eq:standard-deviation}
\end{equation}
Where $R$ is divided into sections between each visit to $s$ and $X$ is a list of the sum of weights within these
sections.\\
$\mu$ is the mean cumulative weight of each $x_i$.

\subsection{Constraints}\label{subsec:problem-constraints}
A valid solution must satisfy the following constraints:
\begin{itemize}
    \item The route must visit every node $v \in \{V \setminus s\}$ exactly once:\\
    $\forall_{v \in \{V \setminus s\}}, |\{i \in \{1, \mathellipsis, n\}: r_i = v\}| = 1$
    \item The route must visit $s$ exactly $d$ times:\\
    $|\{i \in \{1, \mathellipsis, n\}: r_i = s\}| = d$
    \item The route must not visit $s$ consecutively:\\
    $\forall_{i \in \{1, \mathellipsis, n-1\}}, r_i \neq r_{i+1}$
    \item The route must end at $s$:\\
    $r_n = s$
\end{itemize}


    \pagebreak

    \section{Literature Review}\label{sec:literature-review}
    This literature review aims to explore existing research and approaches to other combinatorial optimization problems.
There is extensive previous research on various combinatorial problems, for example, the Travelling Salesman
Problem (TSP), Vehicle Routing Problem (VRP) and Tourist Trip Design Problem (TTDP).
It is important to understand how these problems are similar to the one presented in this report, as well as where
those similarities end.
By gaining an understanding of the strengths and limitations of existing approaches to similar problems, we can make
better informed decisions regarding which approaches to investigate, how they may be adapted to suit our specific
constraints and how they might be implemented in practice.
While the approaches taken to these problems may not be directly applicable to our own, it is likely we can adapt their
techniques to suit the specific constraints of this problem.\\

The TSP is one of the most extensively studied combinatorial optimization problems.
While the exact origin of the problem is unclear, it gained populariy

The [problem1] was first defined in [paper1] as [definition], since its introduction [problem1] has been widely
studied, including many variants of the original problem.
blah blah blah.
This is similar to our problem because [reason].
Another problem is [problem1], which is defined by [paper2] as [definition].
blah blah blah.
This is similar to our problem because [reason].
[paper3] proposes a solution to [problem] using [approach].
blah blah blah, pros, cons, etc.\\

READ, YOU NEED TO READ THE PAPERS stop faffing about and read the papers


\subsection{Classical Traveling Salesman Problem (TSP)}
\begin{itemize}
    \item Definition and mathematical formulation
    \item Complexity analysis and NP-hardness
    \item Key solution approaches
    \item Relevance and limitations in relation to our specific problem
\end{itemize}

\noindent\textbf{Recommended Literature:}
\begin{itemize}
    \item Applegate, D. L., Bixby, R. E., Chvátal, V., \& Cook, W. J. (2006). \textit{The Traveling Salesman Problem: A Computational Study}. Princeton University Press.
    \item Laporte, G. (1992). ``The traveling salesman problem: An overview of exact and approximate algorithms.'' \textit{European Journal of Operational Research}, 59(2), 231-247.
    \item Lin, S., \& Kernighan, B. W. (1973). ``An effective heuristic algorithm for the traveling-salesman problem.'' \textit{Operations Research}, 21(2), 498-516.
    \item Helsgaun, K. (2000). ``An effective implementation of the Lin–Kernighan heuristic.'' \textit{European Journal of Operational Research}, 126(1), 106-130.
\end{itemize}

\subsection{Multiple Traveling Salesman Problem (mTSP)}
\begin{itemize}
    \item Extension of the TSP with multiple agents
    \item Mathematical formulation differences from TSP
    \item Application to multi-day planning scenarios
    \item Connection to our requirement of visiting the starting point $d$ times
\end{itemize}

\noindent\textbf{Recommended Literature:}
\begin{itemize}
    \item Bektas, T. (2006). ``The multiple traveling salesman problem: an overview of formulations and solution procedures.'' \textit{Omega}, 34(3), 209-219.
    \item Kara, I., \& Bektas, T. (2006). ``Integer linear programming formulations of multiple salesman problems and its variations.'' \textit{European Journal of Operational Research}, 174(3), 1449-1458.
    \item Gavish, B., \& Srikanth, K. (1986). ``An optimal solution method for large-scale multiple traveling salesmen problems.'' \textit{Operations Research}, 34(5), 698-717.
    \item Carter, A. E., \& Ragsdale, C. T. (2006). ``A new approach to solving the multiple traveling salesperson problem using genetic algorithms.'' \textit{European Journal of Operational Research}, 175(1), 246-257.
\end{itemize}

\subsection{Vehicle Routing Problem (VRP) and Variants}
\begin{itemize}
    \item Basic VRP definition and formulation
    \item Vehicle Routing Problem with Multiple Trips (VRPMT)
    \item Capacitated VRP and other variants
    \item Relevance to our balanced route planning requirement
\end{itemize}

\noindent\textbf{Recommended Literature:}
\begin{itemize}
    \item Toth, P., \& Vigo, D. (Eds.). (2002). \textit{The Vehicle Routing Problem}. SIAM Monographs on Discrete Mathematics and Applications.
    \item Cattaruzza, D., Absi, N., \& Feillet, D. (2016). ``Vehicle routing problems with multiple trips.'' \textit{4OR}, 14(3), 223-259.
    \item Brandão, J., \& Mercer, A. (1997). ``A tabu search algorithm for the multi-trip vehicle routing and scheduling problem.'' \textit{European Journal of Operational Research}, 100(1), 180-191.
    \item Olivera, A., \& Viera, O. (2007). ``Adaptive memory programming for the vehicle routing problem with multiple trips.'' \textit{Computers \& Operations Research}, 34(1), 28-47.
\end{itemize}

\subsection{Tourist Trip Design Problem (TTDP)}
\begin{itemize}
    \item Problem definition focusing on tourist-specific constraints
    \item Time-dependent considerations and point-of-interest selection
    \item Personalization aspects in tourist routing
    \item Direct applicability to our problem's tourism context
\end{itemize}

\noindent\textbf{Recommended Literature:}
\begin{itemize}
    \item Vansteenwegen, P., Souffriau, W., \& Van Oudheusden, D. (2011). ``The orienteering problem: A survey.'' \textit{European Journal of Operational Research}, 209(1), 1-10.
    \item Gavalas, D., Konstantopoulos, C., Mastakas, K., \& Pantziou, G. (2014). ``A survey on algorithmic approaches for solving tourist trip design problems.'' \textit{Journal of Heuristics}, 20(3), 291-328.
    \item Souffriau, W., Vansteenwegen, P., Vanden Berghe, G., \& Van Oudheusden, D. (2013). ``The planning of cycle trips in the province of East Flanders.'' \textit{Omega}, 41(3), 522-531.
    \item Garcia, A., Vansteenwegen, P., Arbelaitz, O., Souffriau, W., \& Linaza, M. T. (2013). ``Integrating public transportation in personalised electronic tourist guides.'' \textit{Computers \& Operations Research}, 40(3), 758-774.
\end{itemize}

\subsection{Multi-Objective Optimization in Routing Problems}
\begin{itemize}
    \item Balancing competing objectives (like total weight vs. variance)
    \item Pareto optimality concepts
    \item Solution approaches for multi-objective routing
    \item Applicability to our dual-objective function
\end{itemize}

\noindent\textbf{Recommended Literature:}
\begin{itemize}
    \item Jozefowiez, N., Semet, F., \& Talbi, E. G. (2008). ``Multi-objective vehicle routing problems.'' \textit{European Journal of Operational Research}, 189(2), 293-309.
    \item Paquete, L., \& Stützle, T. (2006). ``A study of stochastic local search algorithms for the biobjective QAP with correlated flow matrices.'' \textit{European Journal of Operational Research}, 169(3), 943-959.
    \item Laporte, G., Semet, F., Matl, P., \& Voß, S. (2018). ``Multi-objective vehicle routing problem.'' \textit{Operations Research Perspectives}, 5, 50-57.
    \item Coello, C. A. C., Lamont, G. B., \& Van Veldhuizen, D. A. (2007). \textit{Evolutionary Algorithms for Solving Multi-Objective Problems}. Springer.
\end{itemize}

\subsection{Balance-Oriented Routing Problems}
\begin{itemize}
    \item Problems focusing on workload balancing
    \item Min-max objectives in routing
    \item Variance minimization approaches
    \item Connection to our goal of minimizing variance between trips
\end{itemize}

\noindent\textbf{Recommended Literature:}
\begin{itemize}
    \item Jozefowiez, N., Semet, F., \& Talbi, E. G. (2009). ``An evolutionary algorithm for the vehicle routing problem with route balancing.'' \textit{European Journal of Operational Research}, 195(3), 761-769.
    \item Dell'Amico, M., Monaci, M., Pagani, C., \& Vigo, D. (2007). ``Heuristic approaches for the fleet size and mix vehicle routing problem with time windows.'' \textit{Transportation Science}, 41(4), 516-526.
    \item Lee, T. R., \& Ueng, J. H. (1999). ``A study of vehicle routing problems with load-balancing.'' \textit{International Journal of Physical Distribution \& Logistics Management}, 29(10), 646-657.
    \item Liu, R., Xie, X., Augusto, V., \& Rodriguez, C. (2013). ``Heuristic algorithms for a vehicle routing problem with simultaneous delivery and pickup and time windows in home health care.'' \textit{European Journal of Operational Research}, 230(3), 475-486.
\end{itemize}

\subsection{Time-Dependent Routing Problems}
\begin{itemize}
    \item Integration of visit durations into routing decisions
    \item Time windows and scheduling constraints
    \item Solution approaches for time-dependent problems
    \item Relevance to our edge weight definition that incorporates visit duration
\end{itemize}

\noindent\textbf{Recommended Literature:}
\begin{itemize}
    \item Ichoua, S., Gendreau, M., \& Potvin, J. Y. (2003). ``Vehicle dispatching with time-dependent travel times.'' \textit{European Journal of Operational Research}, 144(2), 379-396.
    \item Donati, A. V., Montemanni, R., Casagrande, N., Rizzoli, A. E., \& Gambardella, L. M. (2008). ``Time dependent vehicle routing problem with a multi ant colony system.'' \textit{European Journal of Operational Research}, 185(3), 1174-1191.
    \item Hashimoto, H., Yagiura, M., \& Ibaraki, T. (2008). ``An iterated local search algorithm for the time-dependent vehicle routing problem with time windows.'' \textit{Discrete Optimization}, 5(2), 434-456.
    \item Figliozzi, M. A. (2012). ``The time dependent vehicle routing problem with time windows: Benchmark problems, an efficient solution algorithm, and solution characteristics.'' \textit{Transportation Research Part E: Logistics and Transportation Review}, 48(3), 616-636.
\end{itemize}

\subsection{Synthesis and Research Gaps}
\begin{itemize}
    \item Comparison of problem characteristics across reviewed literature
    \item Key methodological approaches applicable to our problem
    \item Identification of research gaps in addressing our specific problem constraints
    \item Potential directions for adaptation of existing methodologies
\end{itemize}

\noindent\textbf{Recommended Literature:}
\begin{itemize}
    \item Laporte, G. (2009). ``Fifty years of vehicle routing.'' \textit{Transportation Science}, 43(4), 408-416.
    \item Cordeau, J. F., Gendreau, M., Laporte, G., Potvin, J. Y., \& Semet, F. (2002). ``A guide to vehicle routing heuristics.'' \textit{Journal of the Operational Research Society}, 53(5), 512-522.
    \item Eksioglu, B., Vural, A. V., \& Reisman, A. (2009). ``The vehicle routing problem: A taxonomic review.'' \textit{Computers \& Industrial Engineering}, 57(4), 1472-1483.
    \item Vidal, T., Crainic, T. G., Gendreau, M., \& Prins, C. (2013). ``Heuristics for multi-attribute vehicle routing problems: A survey and synthesis.'' \textit{European Journal of Operational Research}, 231(1), 1-21.
\end{itemize}

\subsection{Conclusion}
\begin{itemize}
    \item Summary of most relevant approaches
    \item Recommendation for methodological direction
    \item Justification for selected approach based on literature findings
\end{itemize}

\noindent\textbf{Recommended Literature:}
\begin{itemize}
    \item Gendreau, M., Potvin, J. Y., Bräumlaysy, O., Hasle, G., \& Løkketangen, A. (2008). ``Metaheuristics for the vehicle routing problem and its extensions: A categorized bibliography.'' In \textit{The vehicle routing problem: Latest advances and new challenges} (pp. 143-169). Springer.
    \item Bräysy, O., \& Gendreau, M. (2005). ``Vehicle routing problem with time windows, Part I: Route construction and local search algorithms.'' \textit{Transportation Science}, 39(1), 104-118.
    \item Glover, F., \& Kochenberger, G. A. (Eds.). (2003). \textit{Handbook of Metaheuristics}. Springer.
    \item Talbi, E. G. (2009). \textit{Metaheuristics: From Design to Implementation}. John Wiley \& Sons.
\end{itemize}


    \pagebreak

    \section{Algorithms Investigated and Their Implementation}\label{sec:algorithms-investigated}
    \todo{There should be a link either here or at end of literature which forms the basic for different methods (clustering, routing, trip generation).}
\todo{Paragraph describing different types of algorithm used (Routing then cluster, Cluster then Routing, Genetic, etc.)}
\todo{Remember to justify the choice of algorithms. You may also need to explain how to adopt these algorithms in your work. A figure showing the ralationship between different components of your work may also help.}


\subsection{Clustering}\label{subsec:clustering}
Clustering as a concept can be described as `the unsupervised classification of patterns (observation, data items, or
feature vectors) into groups (clusters)'\todo{cite Data Clustering: A review, A.K. Jain}.
In our problem, clustering will be used to group locations together to form an itinerary for each day of the trip.
These clusters (or days) will then be used as an input for some routing algorithm, which will try and find an
optimal route for each day.
These routes can then be combined to form a complete route for the trip, which can be evaluated using our cost
function.
The goal of our clustering algorithms is to find a set of clusters that, when combined with some routing algorithm,
will produce a route that minimises the cost function.\\
\\
The clustering algorithms implemented in this project are: K-Means, Genetic Clustering and Genetic Centroid-based
Clustering.
\subsubsection{K-Means}\label{subsubsec:k-means}
K-Means is an iterative clustering algorithm that defines its clusters using a set of centroids (means) which are
given a location in the input space.
The algorithm starts by initialising random centroids and iteratively improving the clustering from there, continuing
until the algorithm converges (on a local optimum) or an iteration limit is reached.
While this algorithm may not find the best solution, it is rather quick, with a time complexity of $O(kni)$, where $k$
is the number of clusters, $n$ is the number of locations and $i$ is the maximum allowed number of iterations.\\

\noindent
In our implementation of K-Means we will initialise our centroids by generating random geographic coordinates in
a similar area to the locations in our input.
We will be using the coordinates of our locations to calculate the Euclidean distances between locations and centroids,
these locations will be assigned to the cluster of the closest centroid.

\begin{figure}[H]\label{fig:_assign_nodes_to_centroid}
    \centering
    \includegraphics[width = \textwidth]{clustering-_assign_nodes_to_centroid}
    \caption{Clustering.\_assign\_nodes\_to\_centroid in algorithms\textbackslash clustering.py}
\end{figure}
\noindent
After assignment, the centroids are recalculated such that their coordinates are the average of all locations
assigned to their cluster.

\begin{figure}[H]\label{fig:_compute_means}
    \centering
    \includegraphics[width = \textwidth]{clustering-_compute_means}
    \caption{KMeans.\_compute\_means in algorithms\textbackslash clustering.py}
\end{figure}
\noindent
These steps of cluster assignment and centroid recalculation are repeated until either a maximum allowed number of
iterations is reached, or until the algorithm converges on an optimum solution.
Our convergence criterion is that the centroids stop changing between iterations, i.e., the centroids are the same
as the previous iteration.

\begin{figure}[H]\label{fig:find_clusters}
    \centering
    \includegraphics[width = \textwidth]{clustering-KMeans.find_clusters}
    \caption{KMeans.find\_clusters in algorithms\textbackslash clustering.py}
\end{figure}
\\
\noindent
Below is an example of the K-Means algorithm run on an input of 25 points of interest around London over seven days.

\begin{figure}[H]
    \ContinuedFloat*
    \includegraphics[width = \textwidth]{KMeans_London_Step1}
    \caption{Step1}
\end{figure}
\begin{figure}[H]
    \ContinuedFloat
    \includegraphics[width = \textwidth]{KMeans_London_Step2}
    \caption{Step2}
\end{figure}
\begin{figure}[H]
    \ContinuedFloat
    \includegraphics[width = \textwidth]{KMeans_London_Step3}
    \caption{Step3}
\end{figure}
\begin{figure}[H]
    \ContinuedFloat
    \includegraphics[width = \textwidth]{KMeans_London_Step4}
    \caption{Step4}
\end{figure}

\subsubsection{Genetic Centroid-based Clustering}
\todo{Explain genetic centroid-based clustering and how it differs from general clustering.}
\subsubsection{Genetic Clustering}
\todo{Explain genetic clustering}

\subsection{Routing}\label{subsec:routing}
\todo{Explain purpose of routing/goal of algorithms.}
\subsubsection{Brute Force}\label{subsubsec:brute-force-routing}
\todo{Write brute force explanation}
The brute force algorithm is an exhaustive algorithm that tries every possible route to find one with the least cost.
By checking every route it is guaranteed to find the optimal route, however, its computational cost becomes
impractical as input size grows, having a time complexity of $O(n!)$.\todo{Maybe cite time complexity of brute force?}
Considering we will be comparing algorithms based on their speed and the quality of their results, brute force is a
useful benchmark, providing a lower bound for speed and an upper bound for quality.\\
\\
In our brute force implementation, where n is the number of locations in the route, we will be generating all $n-1!$
permutations of the set $\{1, 2, \mathellipsis, n-1\}$, with each permutation representing the order of locations
visited in a route.
Each route will be evaluated according to our optimisation function, and the route with the lowest cost will be
returned.
We only need to consider $n-1!$ permutations because all our routes will start and end at the same location.\\


\subsubsection{Greedy Routing}\label{subsubsec:greedy-routing}
\todo{Explain greedy routing algorithm}
Greedy routing
\subsubsection{Gift Wrapping}\label{subsubsec:gift-wrapping}
\todo{Explain gift wrapping algorithm}
\todo{Something like: "Once gift wrapping has found a convex hull, a greedy insertion algorithm is used to find the optimal route within the convex hull."}
\subsubsection{Genetic Routing}
\todo{Explain genetic routing}

\subsection{Route Insertion}\label{subsec:route-insertion}
\todo{Explain route insertion, how it is used in route planning and the goal of our algorithms.}
\subsubsection{Brute Force}\label{subsubsec:brute-force-route-insertion}
\todo{Explain how brute force algorithm can be modified for route insertion.}
\subsubsection{Greedy Insertion}\label{subsubsec:greedy-insertion}
\todo{Explain how greedy algorithm can be modified for route insertion.}

\subsection{Trip Generation}\label{subsec:trip-generation}
\todo{Explain trip generation, how it is used in route planning and the goal of our algorithms.}
\subsubsection{Brute Force}\label{subsubsec:brute-force-trip-generation}
\todo{Explain how brute force algorithm can be modified for trip generation.}
\subsubsection{Genetic Trip Generation}
\todo{Explain genetic trip generation}

    \pagebreak

    \section{Algorithm Evaluation and Results}\label{sec:evaluation-and-results}
    This section will contain an explanation of the methodology taken to evaluate different approaches to Multi Day Trip
Planning.
This explanation will include a description of the approaches to be evaluated, alongside the experiment procedure used
and any constraints that were placed on the problem.
Later in this section, the results of these experiments will be presented and discussed.

\subsection{Methodology}\label{subsec:evaluation-methodology}
The purpose of this evaluation is to compare the performance of different approaches to Multi Day Trip Planning.
The experimental process aimed to answer the following questions:
\begin{itemize}
    \item How do the different approaches compare in terms of computation time?
    \item How do the different approaches compare in terms of the quality of trips produced?
    \item How does the performance of each approach scale with input complexity?
\end{itemize}
To answer these questions, each proposed approach will be applied to a range of different inputs, recording the
computation time to produce a trip and the quality of the trip produced according to the cost function described in
section~\ref{subsec:objective-function}.

\noindent
For use in these experiments 25 testing data sets were created, each representing inputs for trips to major cities
around the world.
Each data set includes:
\begin{itemize}
    \item A set of coordinates, representing different points of interest around the city.
    \item A list of durations, representing the time to be spent at each point of interest.
    \item A complete graph, containing the time taken to travel between each point of interest.
\end{itemize}
These were created using openrouteservice, the process of which was previously described in
section~\ref{subsubsec:input-generation} titled ``Input Generation''.
Due to previously discussed limitations of the openrouteservice API, each dataset was limited to 25 points of
interest.\\

\noindent
To investigate how the performance of these algorithms scales with input complexity, experiments with a range of
inputs for the number of locations and number of days in each trip were conducted.
For these experiments, a subset of locations and their corresponding durations and graphs were selected from the 25
datasets, each approach was run on the same subset.
Every approach was run on every dataset for every combination of locations and days.
A full list of the combinations of the number of locations and number of days is available in table~\ref{tab:locations-and-days}.
\begin{table}[H]
    \centering
    \caption{Lists all combinations of the number of locations and number of days used in evaluation.}\label{tab:locations-and-days}
    \begin{tabular}{ r | l }
        Number of & Number of \\
        Locations & Days \\
        25 & 7, 6, 5, 4 \\
        20 & 6, 5, 4, 3 \\
        15 & 5, 4, 3, 2 \\
        10 & 4, 3, 2 \\
        8  & 3, 2 \\
        5  & 2 \\
    \end{tabular}
\end{table}

\noindent
Using various combinations of the implemented clustering, routing and trip generation algorithms, a total of 16
approaches were evaluated.
The full list of these approaches and an explanation of each one, as well as the shorthand used to refer to each
approach, is available in table~\ref{tab:approaches}.
Most of these approaches are either trip generation methods, or different combinations of the clustering and routing
algorithms previously described.
The only exceptions to this are the `Genetic Algorithm Clustering + Greedy Routing + Brute Force' (GAC+GR+BF) and
`Genetic Algorithm Centroid Clustering + Greedy Routing + Brute Force' (GACC+GR+BF) approaches.
These approaches use greedy routing to evaluate the clusters found during the genetic algorithm process.
Then, once evolution has completed, the resultant clusters are passed to Brute Force to find optimal routes within
the clusters found.
The hope is that greedy routing will allow a fast genetic algorithm, while using Brute Force will allow for an
optimal final trip.

\begin{center}
    \caption{Lists and describes all approaches evaluated.}\label{tab:approaches}
    \footnotesize
    \begin{longtable}[H]{| p{6cm} | p{9cm} |}
        \hline
        Approach & Description \\
        \hline
        Genetic Algorithm Trip Generation (GATG) & Generates trip using Genetic Algorithm \\
        \hline
        Greedy Insertion Trip Generation (GITG) & Generates trip using Greedy Insertion \\
        \hline
        K-Means Clustering + Greedy Routing (KM+GR) & Finds clusters using K-Means then finds intra-cluster routes
        Greedy Routing \\
        \hline
        K-Means Clustering + Greedy Insertion (KM+GI) & Finds clusters using K-Means then finds intra-cluster routes
        using Greedy Insertion Routing \\
        \hline
        K-Means Clustering + Brute Force (KM+BF) & Finds clusters using K-Means then finds intra-cluster routes using
        Brute Force Routing \\
        \hline
        K-Means Clustering + Convex Hull (KM+CH) & Finds clusters using K-Means then finds intra-cluster routes using
        Convex Hull Routing \\
        \hline
        K-Means Clustering + Genetic Algorithm Routing (KM+GAR) & Finds clusters using K-Means then finds intra-cluster
        routes using Genetic Algorithm Routing \\
        \hline
        Genetic Algorithm Clustering + Greedy Routing (GAC+GR) & Finds clusters using Genetic Clustering then finds
        intra-cluster routes using Greedy Routing \\
        \hline
        Genetic Algorithm Clustering + Greedy Routing + Brute Force (GAC+GR+BF) & Finds clusters using Genetic
        Clustering then finds intra-cluster routes using Greedy Routing - Once final clusters are obtained,
        Brute Force is used to find final routes \\
        \hline
        Genetic Algorithm Clustering + Greedy Insertion (GAC+GI) & Finds clusters using Genetic Clustering then
        finds intra-cluster routes using Greedy Insertion \\
        \hline
        Genetic Algorithm Centroid Clustering + Greedy Routing (GACC+GR) & Finds clusters using Genetic Centroid
        Clustering then finds intra-cluster routes using Greedy Routing \\
        \hline
        Genetic Algorithm Centroid Clustering + Greedy Routing + Brute Force (GACC+GR+BF) & Finds clusters using Genetic
        Centroid Clustering then finds intra-cluster routes using Greedy Routing - Once final clusters are obtained,
        Brute Force is used to find final routes \\
        \hline
        Genetic Algorithm Centroid Clustering + Greedy Insertion (GACC+GI) & Finds clusters using Genetic Centroid
        Clustering then finds intra-cluster routes using Greedy Insertion Routing \\
        \hline
        Greedy Routing + Greedy Insertion (GR+GI) & Finds route using Greedy Routing then splits route into a multi-day
        trip using Greedy Insertion \\
        \hline
        Convex Hull + Greedy Insertion (CH+GI) & Finds route using Convex Hull Routing then splits route into a
        multi-day trip using Greedy Insertion \\
        \hline
        Genetic Algorithm Routing + Greedy Insertion (GAR+GI) & Finds route using Genetic Algorithm Routing then splits
        route into a multi-day trip using Greedy Insertion \\
        \hline
    \end{longtable}
\end{center}

During these experiments, genetic algorithms all shared the following hyperparameters:
\begin{itemize}
    \item Number of Generations: 150,
    \item Population Size: 50,
    \item Crossover Rate: 0.9.
\end{itemize}
Genetic Clustering approaches used a mutation rate of 0.1, while Genetic Routing approaches used a mutation rate of 0.4.

\subsection{Results \& Analysis}\label{subsec:results-and-analysis}
Through this experimentation each of the 25 datasets were tested using each combination of locations and days,
resulting in a total of 450 inputs given to each approach.
The average results of tests ran for several different combinations of locations and days are shown in
tables~\ref{tab:averages-8-locations-3-days} (8 locations and 3 days),
~\ref{tab:averages-15-locations-4-days} (15 locations and 4 days)
and~\ref{tab:averages-25-locations-7-days} (25 locations and 7 days).
The computation time and evaluation shown for each approach is the average of values obtained from all datasets ran
with these location and day inputs.
Also included are the standard deviation ($\sigma$), which represents the deviation in results between different
location data sets, and the coefficient of variation ($\frac{\sigma}{\mu}$), which represents the ratio between mean
values and their standard deviation.
The results are sorted by evaluation, and the best and worst values in each row are highlighted in bold and italic
respectively.
Table~\ref{tab:averages-all-tests} shows average results across all data sets and input sizes.
\begin{table}[H]
    \centering
    \caption{Lists average computation time and evaluation for each approach across tests using 8 locations and 3 days. This table also includes evaluations for Brute Force Trip Generation (BFTG) and Brute Force + Greedy Insertion (BF+GI)}\label{tab:averages-8-locations-3-days}
    \scriptsize
    \begin{tabular}{lllllll}
        Approach   & Computation  & Computation & Computation & Evaluation - $\mu$ & Evaluation - $\sigma$ & Evaluation - $\frac{\sigma}{\mu}$ \\
                   & Time - $\mu$ & Time - $\sigma$ & Time - $\frac{\sigma}{\mu}$ & & & \\
        BFTG       & \textit{10.963727}                                                  & \textit{1.373076}                                                     & 0.125238083                                                                        & \textbf{580.843119} & \textbf{145.607101}   & 0.25068232                        \\
        GAC+GR+BF  & 2.652661                                                            & 0.924027                                                              & 0.348339523                                                                        & 581.771109          & 147.287947            & 0.253171643                       \\
        GATG       & 0.828975                                                            & 0.470336                                                              & 0.56737038                                                                         & 582.483376          & 146.081298            & 0.250790502                       \\
        GAC+GR     & 2.652392                                                            & 0.924045                                                              & 0.348381659                                                                        & 583.209066          & 148.632858            & 0.254853476                       \\
        GACC+GR+BF & 1.901471                                                            & 0.515371                                                              & 0.271037872                                                                        & 592.071590          & 146.909233            & 0.248127482                       \\
        GACC+GR    & 1.901226                                                            & 0.515358                                                              & 0.271065947                                                                        & 593.390534          & 148.322398            & 0.249957472                       \\
        GAR+GI     & 0.691986                                                            & 0.224216                                                              & 0.324017778                                                                        & 650.987103          & 158.635774            & 0.243684972                       \\
        BF+GI      & 0.300547                                                            & 0.055220                                                              & 0.183732569                                                                        & 651.094458          & 160.621126            & 0.246694046                       \\
        CH+GI      & 0.003120                                                            & 0.004787                                                              & \textit{1.534394689}                                                               & 657.864674          & 157.007055            & 0.238661629                       \\
        GITG       & 0.004070                                                            & 0.004507                                                              & 1.107453904                                                                        & 661.532642          & 173.934502            & 0.26292656                        \\
        GR+GI      & \textbf{0.000024}                                                   & \textbf{0.000003}                                                     & \textbf{0.121703435}                                                               & 666.328613          & 176.682224            & \textit{0.265157792}              \\
        KM+GAR     & 1.287709                                                            & 0.576568                                                              & 0.447747083                                                                        & 670.487638          & 156.321777            & 0.233146398                       \\
        KM+BF      & 0.013186                                                            & 0.002541                                                              & 0.192707762                                                                        & 670.653112          & 156.392637            & 0.233194529                       \\
        KM+GR      & 0.011910                                                            & 0.001995                                                              & 0.167545233                                                                        & 672.931430          & 159.437076            & 0.236929156                       \\
        KM+CH      & 0.013184                                                            & 0.002048                                                              & 0.155312695                                                                        & 674.160886          & 157.178385            & 0.233146699                       \\
        GAC+GI     & 4.552093                                                            & 1.227828                                                              & 0.269728177                                                                        & 969.678679          & 214.733237            & 0.221447828                       \\
        GACC+GI    & 3.908192                                                            & 0.733851                                                              & 0.187772562                                                                        & 970.224031          & 213.070291            & \textbf{0.219609373}              \\
        KM+GI      & 0.012178                                                            & 0.002108                                                              & 0.173063036                                                                        & \textit{996.233150} & \textit{235.638250}   & 0.23652922
    \end{tabular}
\end{table}
\begin{table}[H]
    \centering
    \caption{Lists average computation time and evaluation for each approach across tests using 15 locations and 4 days}\label{tab:averages-15-locations-4-days}
    \scriptsize
    \begin{tabular}{lllllll}
        Approach   & Computation  & Computation & Computation & Evaluation - $\mu$ & Evaluation - $\sigma$ & Evaluation - $\frac{\sigma}{\mu}$ \\
                   & Time - $\mu$ & Time - $\sigma$ & Time - $\frac{\sigma}{\mu}$ & & & \\
        GAC+GR+BF  & 2.390620                                                            & 0.457114                                                              & 0.191211464                                                                        & \textbf{929.924224}  & 129.695911            & 0.139469333                       \\
        GATG       & 0.999130                                                            & 0.155427                                                              & 0.155562286                                                                        & 931.678599           & 131.730899            & 0.141390925                       \\
        GAC+GR     & 2.387176                                                            & 0.456494                                                              & 0.191227802                                                                        & 932.052914           & 129.772740            & 0.139233233                       \\
        GACC+GR+BF & 2.160390                                                            & 0.434917                                                              & 0.201313923                                                                        & 941.841538           & 130.008533            & 0.138036524                       \\
        GACC+GR    & 2.143543                                                            & 0.434804                                                              & 0.202843494                                                                        & 944.666649           & \textbf{129.641948}   & 0.137235657                       \\
        GITG       & 0.012276                                                            & 0.003757                                                              & 0.306096212                                                                        & 952.964413           & 132.694472            & 0.13924389                        \\
        CH+GI      & 0.008618                                                            & 0.002095                                                              & 0.243065556                                                                        & 959.980725           & 144.641551            & \textit{0.150671308}              \\
        GR+GI      & \textbf{0.000039}                                                   & \textbf{0.000004}                                                     & \textbf{0.104597412}                                                               & 961.133817           & 135.196293            & 0.14066334                        \\
        GAR+GI     & 0.939494                                                            & 0.159025                                                              & 0.169266722                                                                        & 963.440913           & 143.976282            & 0.14943966                        \\
        KM+BF      & \textit{7.332659}                                                   & \textit{31.810776}                                                    & \textit{4.338231859}                                                               & 1113.188018          & 159.945485            & 0.143682363                       \\
        KM+GAR     & 1.684152                                                            & 0.334373                                                              & 0.198540614                                                                        & 1113.690387          & 160.192210            & 0.143839088                       \\
        KM+GR      & 0.013239                                                            & 0.004664                                                              & 0.352271037                                                                        & 1117.856057          & 159.415227            & 0.142608009                       \\
        KM+CH      & 0.016252                                                            & 0.004993                                                              & 0.307194959                                                                        & 1121.260628          & 161.795389            & 0.144297753                       \\
        GACC+GI    & 6.847682                                                            & 0.904047                                                              & 0.132022276                                                                        & 1647.327226          & 225.550233            & \textbf{0.1369189}                \\
        GAC+GI     & 6.735855                                                            & 0.892049                                                              & 0.132432856                                                                        & 1654.194200          & 230.241480            & 0.139186487                       \\
        KM+GI      & 0.013876                                                            & 0.004755                                                              & 0.342677914                                                                        & \textit{1680.691761} & \textit{232.201145}   & 0.138158079
    \end{tabular}
\end{table}
\begin{table}[H]
    \centering
    \caption{Lists average computation time and evaluation for each approach across tests using 25 locations and 7 days}\label{tab:averages-25-locations-7-days}
    \scriptsize
    \begin{tabular}{lllllll}
        Approach   & Computation  & Computation & Computation & Evaluation - $\mu$ & Evaluation - $\sigma$ & Evaluation - $\frac{\sigma}{\mu}$ \\
                   & Time - $\mu$ & Time - $\sigma$ & Time - $\frac{\sigma}{\mu}$ & & & \\
        GAC+GR+BF  & 2.757364                                                            & 0.070118                                                              & 0.025429523                                                                        & \textbf{1224.921655} & 218.453597            & 0.178340873                       \\
        GAC+GR     & 2.752229                                                            & 0.069379                                                              & \textbf{0.025208386}                                                               & 1227.974831          & 219.077690            & 0.178405684                       \\
        GATG       & 1.174132                                                            & 0.065157                                                              & 0.055493493                                                                        & 1249.211698          & 232.957427            & 0.186483546                       \\
        GAR+GI     & 1.145115                                                            & 0.042931                                                              & 0.037490438                                                                        & 1298.771994          & 217.955721            & 0.16781677                        \\
        GITG       & 0.037008                                                            & 0.004152                                                              & 0.112203647                                                                        & 1299.978163          & 229.402565            & 0.176466476                       \\
        GACC+GR+BF & 2.322438                                                            & 0.354714                                                              & 0.152733636                                                                        & 1300.039380          & 256.373114            & 0.197204114                       \\
        GR+GI      & \textbf{0.000055}                                                   & \textbf{0.000010}                                                     & 0.174405584                                                                        & 1303.278191          & 224.211835            & 0.172036819                       \\
        GACC+GR    & 2.171517                                                            & 0.188289                                                              & 0.086708524                                                                        & 1303.379124          & 256.455549            & 0.196762051                       \\
        CH+GI      & 0.023478                                                            & 0.003493                                                              & 0.148782618                                                                        & 1303.547054          & \textbf{216.694963}   & 0.166234861                       \\
        KM+GAR     & 2.473922                                                            & 0.134556                                                              & 0.054389867                                                                        & 1423.636393          & 271.047180            & 0.190390735                       \\
        KM+BF      & 1.697500                                                            & \textit{4.332882}                                                     & \textit{2.55250715}                                                                & 1427.268910          & 282.173377            & \textit{0.197701621}              \\
        KM+GR      & 0.011821                                                            & 0.001488                                                              & 0.125869635                                                                        & 1428.105419          & 272.554303            & 0.190850269                       \\
        KM+CH      & 0.016291                                                            & 0.001898                                                              & 0.116514653                                                                        & 1434.496055          & 272.688511            & 0.190093594                       \\
        GACC+GI    & \textit{11.164482}                                                  & 1.662562                                                              & 0.148915268                                                                        & 2200.445359          & 321.745265            & \textbf{0.146218248}              \\
        GAC+GI     & 9.161882                                                            & 0.358560                                                              & 0.039136028                                                                        & 2233.113133          & 344.559348            & 0.154295518                       \\
        KM+GI      & 0.012839                                                            & 0.001507                                                              & 0.117347705                                                                        & \textit{2261.413558} & \textit{383.340894}   & 0.169513839
    \end{tabular}
\end{table}
\begin{table}[H]
    \centering
    \caption{Lists average computation time and evaluation for each approach across all tests.}
    \label{tab:averages-all-tests}
    \scriptsize
    \begin{tabular}{lllllll}
        Approach   & Computation  & Computation & Computation & Evaluation - $\mu$ & Evaluation - $\sigma$ & Evaluation - $\frac{\sigma}{\mu}$ \\
                   & Time - $\mu$ & Time - $\sigma$ & Time - $\frac{\sigma}{\mu}$ & & & \\
        GAC+GR+BF  & 2.979977                 & 7.530829                    & 2.527143137                             & \textbf{1020.368065} & \textbf{375.291204} & 0.367799833                       \\
        GAC+GR     & 2.449077                 & 0.573792                    & \textbf{0.234288951}                    & 1022.932742        & 376.517470            & 0.368076467                       \\
        GATG       & 1.056100                 & 0.350635                    & 0.332009669                             & 1025.823776        & 377.272585            & 0.367775239                       \\
        GACC+GR+BF & 3.283957                 & 11.213699                   & 3.414691397                             & 1026.018469        & 378.328927            & 0.368735007                       \\
        GACC+GR    & 2.058841                 & 0.552098                    & 0.268159607                             & 1035.318311        & 384.760096            & 0.37163459                        \\
        GITG       & 0.016816                 & 0.013973                    & 0.830951736                             & 1097.052085        & 404.128161            & 0.368376458                       \\
        GAR+GI     & 0.979447                 & 0.291409                    & 0.297524026                             & 1098.753750        & 405.460659            & 0.36901868                        \\
        GR+GI      & \textbf{0.000053}        & \textbf{0.000078}           & 1.472593046                             & 1101.631357        & 403.964025            & 0.366696193                       \\
        CH+GI      & 0.012669                 & 0.010609                    & 0.837414977                             & 1101.948744        & 405.566210            & 0.368044532                       \\
        KM+BF      & \textit{9.931990}        & \textit{35.391964}          & \textit{3.563431295}                    & 1142.732633        & 418.983039            & \textbf{0.366650104}              \\
        KM+GAR     & 1.691711                 & 0.608345                    & 0.359603359                             & 1232.529265        & 461.950473            & 0.374798787                       \\
        KM+GR      & 0.013553                 & 0.005196                    & 0.383363287                             & 1236.909841        & 463.293563            & 0.374557261                       \\
        KM+CH      & 0.017589                 & 0.006696                    & 0.380662466                             & 1243.418740        & 466.350523            & 0.375055087                       \\
        GACC+GI    & 8.228977                 & 3.850029                    & 0.467862454                             & 1792.357060        & 674.690213            & 0.376426231                       \\
        GAC+GI     & 7.738760                 & 3.107706                    & 0.401576755                             & 1802.357683        & 686.223849            & 0.380736773                       \\
        KM+GI      & 0.014386                 & 0.005420                    & 0.376779259                             & \textit{1830.032473} & \textit{701.160152} & \textit{0.383140825}
    \end{tabular}
\end{table}
From analysing these results, there are a few approaches that stand out as being particularly effective.
GAC+GR+BF consistently produces some of the best evaluations across all datasets, while also being relatively fast.
In fact, approaches that utilise genetic algorithms for clustering or trip generation overwhelmingly outperform the
conventional heuristic approaches; on every input with locations greater than 10, genetic algorithms held the top
four spots for route evaluation.
Other approaches of note are GITG, GR+GI and CH+GI, which all find trips within a few hundredths of a second, while
remaining middle of the pack in terms of trip quality.

By far the worst performing approaches are GACC+GI, GAC+GI and KM+GI\@.
The quality of routes produced by the approaches that perform Greedy Insertion Routing are significant outliers amongst
the other approaches, which begs the question as to why this is so.
Logically, one would expect greedy insertion routing to produce a route at least as good and greedy routing, if not better.
It is suggested that further investigation is needed to understand why this is the case, and whether this could be
caused by a
flawed implementation of the greedy insertion algorithm.

Amongst algorithms that start by clustering locations, K-Means Clustering appears to produce significantly worse
routes than those found with genetic approaches.
K-Means approaches are much faster than their genetic counterparts, but this trade off doesn't appear to be worth it
considering that Routing+Insertion approaches are even faster and produce better routes.
Even the K-Means inspired genetic algorithms are outperformed by the standard genetic algorithm approaches,
indicating that these geographic heuristics for clustering do not translate well to producing a good trip.

Further evidence of this is how Convex Hull Routing produces the worst results out of Routing+Insertion approaches.
Convex Hull Routing was included in this project to investigate whether a convex hull could provide a good starting
point for greedy insertion, hopefully being able to speed up the process or produce better results.
While convex hull routing does produce slightly worse results though, it did find routes around 25\% faster than
Greedy Insertion Trip Generation.

It is also worth noting that these geometric based heuristics often produced large coefficients of variation,
indicating that compared to other approaches, different location inputs resulted in a larger difference in the quality
of the routes produced.

The full set of experiment results is available in this project's related gitlab repository, located in
`data\textbackslash results.csv'.

    \pagebreak

    \section{Conclusion and Future Work}\label{sec:conclusion-and-future-work}
    The goal of this project was to investigate and compare different approaches to solving the proposed multi-day route
planning problem.
Throughout this project, a number of algorithms were investigated, implemented and evaluated to determine their
effectiveness.
From the results shown in section~\ref{subsec:results-and-analysis}, a conclusion can be drawn that the most
promising approaches are those that first cluster locations, before finding routes between them.
Genetic clustering methods proved particularly exemplary, often finding the best trips out of the approaches
tested, while still maintaining a relatively fast computation speed.

While the aims of this project have largely been accomplished, further study is required to resolve gaps in the
potential solutions considered.
Certain categories of approach are somewhat lacking in the diversity of algorithms implemented; clustering for
example, only includes one conventional heuristic approach in the form of k-means, with other clustering algorithms
being the meta-heuristic genetic algorithm approaches.
Most notable is greedy insertion being the only implemented insertion algorithm.
While Greedy Insertion proved effective when combined with a routing algorithm, it would have been interesting to
have other insertion algorithms to compare it to.

Future work on this problem should investigate additional clustering algorithms, of particular
recommendation would be researching spectral clustering, a clustering technique that aims to group together graph
nodes that are closely connected.
It would also likely be of value to examine further local search techniques such as 2-opt that, similar to greedy
insertion, could potentially be implemented to create multi-day trips from pre-existing TSP routes.
Finally, considering the success of genetic algorithms in this project, it could be beneficial to consider other
meta-heuristic approaches such as Ant Colony Optimisation or Neural Network approaches.

    \pagebreak

    \section{Bibliography}\label{sec:bibliography}
    \printbibliography
\end{document}