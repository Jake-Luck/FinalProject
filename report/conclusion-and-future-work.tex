The goal of this project was to investigate and compare different approaches to solving the proposed multi-day route
planning problem.
Throughout this project, a number of algorithms were investigated, implemented and evaluated to determine their
effectiveness.
From the results shown in section~\ref{subsec:results-and-analysis}, a conclusion can be drawn that the most
promising approaches are those that first cluster locations, before finding routes between them.
Genetic clustering methods proved particularly exemplary, often finding the best trips out of the approaches
tested, while still maintaining a relatively fast computation speed.

While the aims of this project have largely been accomplished, further study is required to resolve gaps in the
potential solutions considered.
Certain categories of approach are somewhat lacking in the diversity of algorithms implemented; clustering for
example, only includes one conventional heuristic approach in the form of k-means, with other clustering algorithms
being the meta-heuristic genetic algorithm approaches.
Most notable is greedy insertion being the only implemented insertion algorithm.
While Greedy Insertion proved effective when combined with a routing algorithm, it would have been interesting to
have other insertion algorithms to compare it to.

Future work on this problem should investigate additional clustering algorithms, of particular
recommendation would be researching spectral clustering, a clustering technique that aims to group together graph
nodes that are closely connected.
It would also likely be of value to examine further local search techniques such as 2-opt that, similar to greedy
insertion, could potentially be implemented to create multi-day trips from pre-existing TSP routes.
Finally, considering the success of genetic algorithms in this project, it could be beneficial to consider other
meta-heuristic approaches such as Ant Colony Optimisation or Neural Network approaches.